%%%%%%%%%%%%%%%%%%%%%%%%%%%%%%%%%%%%%%%%%%%%%%%%%%%%%%%%%%%%%%%%%%%%%%%%%%%%%%%%%%%%%%%%%%%%%%%%%%%%%%%%%%%%%%%%%%
% File: Considerazioni.tex
% Current Version: 0.1.0
% Created: 2013-12-09
% Author: Martignago Jimmy
% Email: jimmy.martignago@gmail.com
%----------------------------------------------------------------------------------------------------------------
% Modification History:
% Version		Author	 			Date			Action
% 0.1.0			Martignago Jimmy	2013-12-14		Apportate modifiche a seguito verifica.
% 0.0.3			Martignago Jimmy	2013-12-12		Modifica errori impaginazione.
% 0.0.2			Martignago Jimmy	2013-12-11		Stesura file.
% 0.0.1			Martignago Jimmy	2013-12-09		Creazione file ed impostazione struttura.
%%%%%%%%%%%%%%%%%%%%%%%%%%%%%%%%%%%%%%%%%%%%%%%%%%%%%%%%%%%%%%%%%%%%%%%%%%%%%%%%%%%%%%%%%%%%%%%%%%%%%%%%%%%%%%%%%%
\section{Considerazioni finali}
\label{considerazioni}
I variegati domini applicativi e tecnologici espressi nei capitolati, hanno portato ad una fervente discussione tra i membri del gruppo. Riteniamo che questa prima fase sia stata molto utile, dato che ha permesso di definire con chiarezza quali erano gli obiettivi e gli interessi dei componenti. Di fondamentale aiuto, è stato il fatto di dover scegliere tra progetti completamente diversi tra loro e possiamo supporre sia stato gradito anche da altri colleghi.
\\Prima di costituire ufficialmente il gruppo, alcuni dei membri hanno sfruttato la presentazione dei capitolati, avvenuta in data 2013-10-31, per chiarire alcuni punti delle varie presentazioni. Successivamente sono stati presi contatti con alcuni proponenti, chiedendo di avere un incontro esplorativo per approfondire i domini tecnologici.
\\Dopo la costituzione del gruppo, si è passati ad una prima discussione informale tra i componenti. Lo scopo stava nel comporre un quadro della situazione considerando gli interessi individuali, la spendibilità delle conoscenze che si sarebbero acquisite e la gestione rischi insiti nei capitolati. In questa fase si è deciso di scartare i capitolati C4 (rif. \ref{capitolato4}) e C5 (rif. \ref{capitolato5}) perché alla luce delle considerazioni fatte, nessun componente del gruppo esprimeva interesse verso di loro.
\\In ultima analisi, dopo alcuni confronti con i proponenti, si è proceduto per votazione sostanzialmente formale, dato che l'interesse della maggioranza del gruppo convergeva sul capitolato C3 (rif. \ref{capitolato3}).
\\Altri dettagli riguardanti questa discussione, sono contenuti nel documento \textit{Verbale1 interno 2013-11-27}. 