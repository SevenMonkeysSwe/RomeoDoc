\section{Requisiti}
\label{requisiti}
Di seguito vengono riportati tutti i requisiti derivanti dal capitolato, da casi d'uso\glossario{}, da incontri con i proponenti, oppure da necessità interne. I requisiti vengono presentati per mezzo di quattro tabelle distinte, ognuna delle quali rappresenta una diversa categoria di requisiti. Ogni tabella contiene quattro colonne che indicano, per ogni requisito, i seguenti attributi:
\begin{itemize}
\item \textbf{Requisito:} codice identificativo univoco del requisito;
\item \textbf{Tipologia:} tipo e importanza del requisito;
\item \textbf{Descrizione:} breve descrizione del requisito;
\item \textbf{Fonti:} fonti dalle quali deriva il requisito.
\end{itemize}

I requisiti sono identificati univocamente per mezzo di questa codifica:

\begin{center}
R[importanza][tipo][codice].
\end{center}
\begin{itemize}
\item \textit{importanza} può assumere i seguenti valori:
	
\begin{itemize}
\item 0: Requisito obbligatorio;
\item 1: Requisito desiderabile; 
\item 2: Requisito opzionale.
\end{itemize}

\item \textit{tipo} può assumere i seguenti valori:

\begin{itemize}
\item F: Funzionale;
\item Q: Di qualità;
\item P: Prestazionale;
\item V: Vincolo.
\end{itemize}

\item \textit{codice} è l'identificativo numerico univoco, che cataloga i requisiti in modo gerarchico.
\end{itemize}






