\Huge P
\rhead{P}
\normalsize
\begin{itemize}
\item\textbf{Package:} in italiano "pacchetto" è un meccanismo utilizzato nei linguaggi di programmazione per riunire classi logicamente correlate o che forniscono servizi simili.
\item\textbf{PDF (\textit{Portable Document Format}):} è un formato di file basato su un linguaggio di descrizione di pagina sviluppato da Adobe System, per rappresentare documenti in modo indipendente dall'hardware e dal software utilizzato per generarli o visualizzarli.

\item\textbf{PERT:} vedi Diagramma di PERT\glossario{}.

\item\textbf{PET (\textit{Positron Emission Tomography}):} è una tecnica di medicina nucleare e di diagnostica medica, utilizzata per la produzione di immagini del corpo. La PET fornisce informazioni di tipo fisiologico; con essa si possono ottenere mappe dei processi funzionali all'interno del corpo.

\item\textbf{Pixel:} elemento puntiforme che compone la rappresentazione di un immagine \textit{bitmap}\glossario{}. Ciascun pixel\glossario{} rappresenta il più piccolo elemento dell'immagine ed è caratterizzato dalla propria posizione e da valori quali colore ed intensità.

\item\textbf{PMOD:} software proprietario per l'elaborazione di immagini biomediche.

\item\textbf{PNG (\textit{Portable Network Graphics}):} è un formato dati utilizzato per la rappresentazione di immagini \textit{bitmap}\glossario{}, la cui estensione è \verb!.PNG!. Tale formato utilizza un algoritmo di compressione senza perdita di informazione. Questo significa che, sebbene l'immagine occupi molto meno spazio di quanto dovrebbe, avendo tolto dei bit informativi, la sua qualità percepita dall'occhio umano, non cambia.
\\PNG\glossario{} è stato progettato per trasferire immagini in Internet ed oggigiorno è il formato più utilizzato nella rete.

\item\textbf{Profiling:} è una forma di analisi dinamica del software che misura, per esempio, lo spazio (memoria) o la complessità di un programma, l'uso di una particolare istruzione o la frequenza e durata di una chiamata di funzione. L'uso più comune di questa analisi è nell'ottimizzazione del software.

\item\textbf{Protocol:} combinazione di una serie di feature extractors\glossario{} (in un numero variabile da 0 a N) e da un algoritmo di clustering\glossario{}.

\item\textbf{Protocollo:} vedi Protocol\glossario{}.
\end{itemize}
\pagebreak

%%%%%%%%%%%%%%%%%%%%%%%%%%%%%%%%%%%%%%%%%%%%%%%%%%%%%%%%%%%%%%
%QQQQQQQQQQQQQQQQQQQQQQQQQQQQQQQQQQQQQQQQQQQQQQQQQQQQQQQQQQQQQ
%%%%%%%%%%%%%%%%%%%%%%%%%%%%%%%%%%%%%%%%%%%%%%%%%%%%%%%%%%%%%%
\Huge Q
\rhead{Q}
\normalsize
\begin{itemize}
\item\textbf{Qt:} framework multipiattaforma, ampiamente utilizzato per lo sviluppo di applicazioni software con interfaccia grafica. Usa il linguaggio C++\glossario{} standard, ma fa un largo uso di uno speciale generatore di codice chiamato \textbf{moc} (\textit{Meta Object Compiler}), che si posiziona ad un livello intermedio tra codice sorgente e codice macchina.
\end{itemize}
\pagebreak

%%%%%%%%%%%%%%%%%%%%%%%%%%%%%%%%%%%%%%%%%%%%%%%%%%%%%%%%%%%%%%
%RRRRRRRRRRRRRRRRRRRRRRRRRRRRRRRRRRRRRRRRRRRRRRRRRRRRRRRRRRRRR
%%%%%%%%%%%%%%%%%%%%%%%%%%%%%%%%%%%%%%%%%%%%%%%%%%%%%%%%%%%%%%
\Huge R
\rhead{R}
\normalsize
\begin{itemize}
\item\textbf{Repository:} database in grado di contenere svariate tipologie di dati, corredate da relative informazioni (metadati). Offre inoltre un sistema di versionamento in grado di tener traccia delle modifiche effettuate al suo interno. Generalmente condiviso da più utenti, ognuno in grado di accedervi autonomamente per apportare modifiche. \`E implicitamente un servizio di condivisione dati.

\item\textbf{RTTI \textit{(RunTime Type Information}):} è un meccanismo del linguaggio di programmazione C++\glossario{}, che consente di recuperare informazioni riguardanti il tipo di un oggetto a \textbf{runtime}. RTTI può essere applicato si a tipi primitivi, sia a tipi derivati.
\end{itemize}
\pagebreak

%%%%%%%%%%%%%%%%%%%%%%%%%%%%%%%%%%%%%%%%%%%%%%%%%%%%%%%%%%%%%%
%SSSSSSSSSSSSSSSSSSSSSSSSSSSSSSSSSSSSSSSSSSSSSSSSSSSSSSSSSSSSS
%%%%%%%%%%%%%%%%%%%%%%%%%%%%%%%%%%%%%%%%%%%%%%%%%%%%%%%%%%%%%%
\Huge S
\rhead{S}
\normalsize
\begin{itemize}
\item\textbf{Schedule Variance:} metrica che indica se si è in linea, in anticipo o in ritardo rispetto alla schedulazione delle attività di progetto pianificate nella baseline.

\item \textbf{Signal:} sono i segnali emessi dai widget Qt\glossario{} a fronte di eventi (interni o esterni).

\item \textbf{Signal \& Slot:} meccanismo con il quale, al verificarsi di un evento, viene emesso un segnale (Signal\glossario{}), al quale è collegato uno o più metodi (Slots\glossario{}) collegati tra di loro tramite una funzione Qt\glossario{}, \emph{connect}.

\item \textbf{Sistema operativo:} spesso abbreviato in \textbf{S.O.}, è un insieme di componenti software, che consente l'utilizzo di varie apparecchiature informatiche da parte di un utente. Esempi di sistemi operativi sono: \textit{Windows}, \textit{Unix}, le distribuzioni \textit{GNU/Linux} e \textit{Mac OS}.

\item \textbf{Slot:} sono identici alle funzioni membro di una classe, con la sola eccezione che può sempre essere collegato a un segnale e quindi sarà invocato ogni volta che un signal\glossario{} verrà emesso. 

\item\textbf{Stub:} porzione di codice che, dati certi input, fornisce sempre gli stessi output predefiniti, al fine di verificare se la funzione chiamante fornisce i risultati attesi dal test.

\item\textbf{Subject:} entità composta da un immagine o da un video e dalla relativa \textit{maschera}\glossario{}. È l'unità fondamentale che viene analizzata da \project{}.

\item\textbf{SV:} si veda Schedule Variance\glossario{}.
\end{itemize}
\pagebreak

%%%%%%%%%%%%%%%%%%%%%%%%%%%%%%%%%%%%%%%%%%%%%%%%%%%%%%%%%%%%%%
%TTTTTTTTTTTTTTTTTTTTTTTTTTTTTTTTTTTTTTTTTTTTTTTTTTTTTTTTTTTTT
%%%%%%%%%%%%%%%%%%%%%%%%%%%%%%%%%%%%%%%%%%%%%%%%%%%%%%%%%%%%%%
\Huge T
\rhead{T}
\normalsize
\begin{itemize}
\item\textbf{TIFF (\textit{Tagged Image File Format}):} è un formato dati utilizzato per la rappresentazione di immagini bitmap\glossario{}, le cui estensioni sono \verb!.TIFF! o \verb!.TIF!. Tale formato nasce per codificare immagini binarie (solo due valori erano possibili per ogni pixel\glossario{}), successivamente ha cominciato a supportare la scala di grigi es infine il multicolore. Oggigiorno TIFF è un formato molto popolare, proprio grazie alla sua profondità di colore, sebbene sia dipendente dal supporto su cui si visualizza l'immagine. In poche parole, una stessa immagine può essere percepita con colori differenti, a seconda dello schermo che si utilizza.

\item\textbf{Time dependent:} dipendente dal tempo. Viene applicata ad immagini dinamiche e si calcola sull'andamento temporale di ogni singolo pixel\glossario{}.

\item \textbf{Tool:} nel linguaggio informatico si tratta di una applicazione che svolge un determinato compito.

\item \textbf{Tooltip:} (letteralmente: consiglio su un oggetto) si intende un comune elemento dell'interfaccia grafica dell'utente. È utilizzato assieme ad un cursore. L'utente passa col cursore sopra un oggetto, senza cliccarlo e appare un piccolo "box" con informazioni supplementari riguardo l'oggetto stesso.
\end{itemize}
\pagebreak