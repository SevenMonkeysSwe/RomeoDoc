%%%%%%%%%%%%%%%%%%%%%%%%%%%%%%%%%%%%%%%%%%%%%%%%%%%%%%%
%KKKKKKKKKKKKKKKKKKKKKKKKKKKKKKKKKKKKKKKKKKKKKKKKKKKKKK
%%%%%%%%%%%%%%%%%%%%%%%%%%%%%%%%%%%%%%%%%%%%%%%%%%%%%%%
\Huge K
\rhead{K}
\normalsize
\begin{itemize}
	\item\textbf{K-means:} algoritmo che divide un insieme di N elementi in K cluster. Il risultato consiste in un vettore di indici, dove per ogni pixel\glossario{}/voxel\glossario{} è riportato il valore del cluster cui il dato è associato (a seconda del codice, può anche ritornare il valore del centroide dei cluster\glossario{} definiti). Ogni dato è associato ad uno e un solo cluster\glossario{}.
\end{itemize}
\pagebreak

%%%%%%%%%%%%%%%%%%%%%%%%%%%%%%%%%%%%%%%%%%%%%%%%%%%%%%%
%LLLLLLLLLLLLLLLLLLLLLLLLLLLLLLLLLLLLLLLLLLLLLLLLLLLLLL
%%%%%%%%%%%%%%%%%%%%%%%%%%%%%%%%%%%%%%%%%%%%%%%%%%%%%%%
\Huge L
\rhead{L}
\normalsize
\begin{itemize}
\item \textbf{Linguaggio di markup:} insieme di regole che descrivono i meccanismi di rappresentazione di un testo che, utilizzando convenzioni standardizzate, sono utilizzabili su più supporti. La tecnica di composizione di un testo con l’uso di marcatori (o espressioni codificate), richiede quindi una serie di convenzioni, ovvero appunto di un linguaggio a marcatori di documenti. Un esempio: \LaTeX.

\item \textbf{Linux:} è una famiglia di sistemi operativi di tipo Unix-like, rilasciati sotto varie possibili distribuzioni, aventi la caratteristica comune di utilizzare come nucleo il kernel Linux. Sviluppato da Linus Torvalds, Progetto GNU e molti altri. Sostenuto dalla Linux Foundation.
\end{itemize}
\pagebreak

%%%%%%%%%%%%%%%%%%%%%%%%%%%%%%%%%%%%%%%%%%%%%%%%%%%%%%%
%MMMMMMMMMMMMMMMMMMMMMMMMMMMMMMMMMMMMMMMMMMMMMMMMMMMMMM
%%%%%%%%%%%%%%%%%%%%%%%%%%%%%%%%%%%%%%%%%%%%%%%%%%%%%%%
\Huge M
\rhead{M}
\normalsize
\begin{itemize}
\item\textbf{Mac OS:} è il sistema operativo di Apple dedicato ai computer Macintosh; il nome è l'acronimo di Macintosh Operating System.

\item\textbf{Mailing list:} è un servizio per la partecipazione di più persone ad una discussione o per la distribuzione di informazioni utili agli interessati attraverso l'invio di email ad una lista di indirizzi di posta elettronica di utenti iscritti.

\item\textbf{Maschera:} immagine statica che funge da delimitatore dell'area di interesse di un immagine principale. Essa ha sempre la stessa dimensione e risoluzione dell'immagine principale, ma contiene solo un bit per ogni pixel\glossario{}. Il bit viene utilizzato come flag: se è 0, il corrispondente pixel\glossario{} dell'immagine originaria non viene considerato, viceversa se è 1, esso viene considerato.

\item\textbf{Matrice di co-occorrenza:} detta anche GLCM (Gray Level Co-occurrence Matrix) è una matrice definita in un intorno di pixel\glossario{} (la finestra di dimensione variabile), che descrive le co-occorrenze di coppie dei livelli di grigio, secondo specificate direzioni. Una matrice GLCM è generata da una distanza e una direzione e ogni elemento della matrice (i,j), rappresenta la probabilità che il pixel [x,y] abbia intensità i e che il pixel successivo abbia intensità j.

\item \textbf{Memory leak:} quando si presenta un consumo non voluto di memoria dovuto alla mancata deallocazione dalla stessa, di variabili/dati non più utilizzati da parte dei processi.

\item \textbf{Milestone:} termine inglese utilizzato nella pianificazione e gestione di progetti complessi per indicare il raggiungimento di obiettivi stabiliti in fase di definizione del progetto stesso. Indica quindi importanti traguardi intermedi nello svolgimento del progetto. Molto spesso sono rappresentate da eventi, cioè da attività con durata zero o di un giorno e vengono evidenziate in maniera diversa dalle altre attività.

\item\textbf{MR (\textit{Magnetic Resonance}):} è una tecnica di medicina nucleare e di diagnostica medica, utilizzata per la produzione di immagini del corpo.

\item \textbf{MVC (\textit{Model View Controller}):} è un design pattern\glossario{} molto diffuso nello sviluppo di sistemi software, in particolare nei sistemi orientati agli oggetti. Prevede la suddivisone del sistema in tre componenti: 
\begin{itemize}
\item \textbf{Model:} rappresenta la logica di \emph{business} dell'applicazione;
\item \textbf{View:} visualizza all'utente i dati dell'applicazione;
\item \textbf{Controller:} riceve i dati dall’utente, aggiorna la View e modifica i dati tramite il Model.
\end{itemize}
\end{itemize}
\pagebreak

%%%%%%%%%%%%%%%%%%%%%%%%%%%%%%%%%%%%%%%%%%%%%%%%%%%%%%%
%NNNNNNNNNNNNNNNNNNNNNNNNNNNNNNNNNNNNNNNNNNNNNNNNNNNNNN
%%%%%%%%%%%%%%%%%%%%%%%%%%%%%%%%%%%%%%%%%%%%%%%%%%%%%%%
\Huge N
\rhead{N}
\normalsize
\begin{itemize}
\item\textbf{Namespace:} in italiano, spazio dei nomi, è una collezione di nomi di entità, definite dal programmatore, omogeneamente usate in uno o più file sorgente. Lo scopo dei namespace è quello di evitare confusione ed equivoci nel caso siano necessarie molte entità con nomi simili, fornendo il modo di raggruppare i nomi per categorie.
\\In particolare, il namespace del c++ è un insieme di nomi in senso matematico, non ha né un ordine né una struttura interna e il programmatore può definirne di suoi.

\item\textbf{NIfTI:} è un formato utilizzato per la rappresentazione e il salvataggio di dati provenienti da esami fMRI\glossario{}. I dati in questione, sono immagini tridimensionali dipendenti o meno dal tempo. Nasce principalmente per cercare di uniformare tutti i formati fino ad allora utilizzati, che avevano portato a grossi problemi di compatibilità. \`E il successore di \textit{Analyze 7.5}\glossario{}, con cui resta compatibile.
\\Un file NIfTI\glossario{} può essere composto in due modi:
\begin{itemize}
\item un singolo file con estensione \verb!.NII! contenente tutte le informazioni necessarie;
\item due files: il primo contenente l'intestazione dell'immagine, cioè tutte le informazioni strutturali (con estensione \verb!.HDR!), il secondo contenente le vere e proprie informazioni riguardo ai \textit{voxel}\glossario{} dell'immagine (con estensione \verb!.IMG!). 
\end{itemize}

\item\textbf{No time dependent:} non dipendente dal tempo. Viene applicata ad immagini statiche e si calcola per ogni singolo pixel\glossario{} all'interno di una finestra. Può essere di:
\begin{itemize}
\item\textbf{Primo ordine:} quando si può calcolare dall'istogramma dell'immagine;
\item\textbf{Secondo ordine:} quando invece si calcolano sui valori della matrice di co-occorrenza\glossario{}.
\end{itemize}

\end{itemize}
\pagebreak

%\Huge O
%\rhead{O}
%\normalsize
%\begin{itemize}

%\end{itemize}
%\pagebreak