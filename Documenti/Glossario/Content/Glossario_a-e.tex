%%%%%%%%%%%%%%%%%%%%%%%%%%%%%%%%%%%%%%%%%%%%%%%%%%%%%%%
%AAAAAAAAAAAAAAAAAAAAAAAAAAAAAAAAAAAAAAAAAAAAAAAAAAAAAA
%%%%%%%%%%%%%%%%%%%%%%%%%%%%%%%%%%%%%%%%%%%%%%%%%%%%%%%
\Huge A
\rhead{A}
\normalsize
\begin{itemize}
\item\textbf{Algoritmo di clustering:} algoritmo che applica la cluster analysis\glossario{} ad una immagine. Esistono diverse tipologie di algoritmi:
\begin{itemize}
\item \textbf{Gerarchici:} algoritmi che prevedono l’aggregazione sequenziale dei Subject\glossario{} in gruppi; tali algoritmi tendono ad essere poco efficienti e a richiedere molte decisioni soggettive da parte dell'utilizzatore. Ricostruiscono l'intera gerarchia dei dati in analisi (albero), in senso ascendente o discendente. Non è noto a priori il numero finale di cluster\glossario{};
\item \textbf{Non gerarchici:} detti di partizione. Algoritmi che prevedono l'aggregazione non sequenziale dei Subject\glossario{} in gruppi, tali algoritmi tendono ad essere più efficienti. È noto a priori il numero di cluster\glossario{} in cui dividere i dati. 
\end{itemize}

\item\textbf{Analyze 7.5:} è un formato dati utilizzato per visualizzare e salvare immagini volumetriche, spesso provenienti da imaging biomedici\glossario{}. Esso è composto da due file:
\begin{itemize}
\item un file contenente i voxel\glossario{} codificati, con estensione \verb!.IMG!;
\item un file header contenente informazioni strutturali, quali dimensione e numero dei voxel\glossario{}, con estensione \verb!.HDR!.
\end{itemize}
Questo tipo di formato sta progressivamente lasciando spazio al più moderno \verb!NIfTI!\glossario{}.

\item\textbf{API (\textit{Application Programming Interface}):} specifica la modalità con cui le varie componenti di un software interagiscono tra di loro. Oltre a specificare come avviene l'accesso ad un database o alle componenti hardware di un computer, spesso le API vengono definite per rendere più agevole lo sviluppo dei componenti delle interfacce grafiche. Nella maggior parte dei casi, una API viene distribuita come una libreria che include le specifiche per routines, strutture dati, classi di oggetti e variabili.

\item\textbf{Aspell:} software libero per il controllo e la correzione ortografica di documenti di testo in vari formati. È disponibile per le piattaforme Unix e Windows\glossario{}, con la possibilità di avere dizionari per più lingue.

\item\textbf{Astah:} software proprietario, utilizzato per modellare diagrammi UML\glossario{}. \`E stato integrato nell'ambiente di sviluppo del gruppo \authorName, nella sua versione \textit{Professional} con licenza per studenti.

\item\textbf{AVI (\textit{Audio Video Interleave}):} è un formato dati contenitore utilizzato per visualizzare e salvare video. Sviluppato da Microsoft, è diventato lo standard video nei sistemi Windows\glossario{}. \verb!AVI! può contenere uno o due flussi audio e un flusso video, ma non supporta alcun formato di sottotitoli.
\end{itemize}
\pagebreak

%%%%%%%%%%%%%%%%%%%%%%%%%%%%%%%%%%%%%%%%%%%%%%%%%%%%%%%
%BBBBBBBBBBBBBBBBBBBBBBBBBBBBBBBBBBBBBBBBBBBBBBBBBBBBBB
%%%%%%%%%%%%%%%%%%%%%%%%%%%%%%%%%%%%%%%%%%%%%%%%%%%%%%%
\Huge B
\rhead{B}
\normalsize
\begin{itemize}
\item\textbf{Bitmap:} immagine composta da una matrice di punti di diverso colore, detta anche mappa di pixel\glossario{}. Ciascun punto viene definito autonomamente rispetto agli altri e viene disposto in una sequenza di righe e di colonne. Un immagine bitmap\glossario{} è caratterizzata da due proprietà:
\begin{itemize}
\item\textbf{risoluzione:} è il numero di pixel\glossario{} contenuti nell'unità di misura considerata. Si misura in PPI (Pixel\glossario{} Per Inch) oppure in DPI (Dot Per Inch);
\item\textbf{profondità:} indica la quantità di memoria che viene dedicata ad ogni pixel\glossario{}, ovvero il numero di bit dedicati al pixel\glossario{} per descriverne il colore e si misura in BPP (Bit Per Pixel\glossario{}).
\end{itemize}

\item\textbf{BMP:} è un formato utilizzato per la rappresentazione di immagini \textit{bitmap}\glossario{}, le cui estensioni sono \verb!.BMP! o meno frequentemente \verb!.DIB!. Questo formato è caratterizzato dal fatto di essere indipendente dal tipo di schermo su cui viene visualizzata l'immagine. La stessa immagine quindi, verrà percepita con le stesse tonalità di colore, su ogni tipo di dispositivo.
\\Il formato BMP è in grado di memorizzare immagini digitali 2D di arbitraria larghezza, altezza e risoluzione. Inoltre, gestisce immagini monocromatiche e multicolore con varie profondità. Attualmente è considerato obsoleto, ma mantiene una buona popolarità grazie al suo utilizzo nell'ambiente Windows\glossario{} per ragioni storiche e di retro-compatibilità.
\item \textbf{Bug:} il termine bug o baco, in informatica, identifica un errore nella scrittura di un programma software.

\item\textbf{Budget Variance:} metrica che indica se alla data corrente si è speso di più o di meno rispetto a quanto previsto a budget alla data corrente.

\item\textbf{BV:} si veda Budget Variance\g{}.
\end{itemize}
\pagebreak

%%%%%%%%%%%%%%%%%%%%%%%%%%%%%%%%%%%%%%%%%%%%%%%%%%%%%%%
%CCCCCCCCCCCCCCCCCCCCCCCCCCCCCCCCCCCCCCCCCCCCCCCCCCCCCC
%%%%%%%%%%%%%%%%%%%%%%%%%%%%%%%%%%%%%%%%%%%%%%%%%%%%%%%
\Huge C
\rhead{C}
\normalsize
\begin{itemize}
\item \textbf{C++:} è un linguaggio di programmazione orientato agli oggetti sviluppato da Bjarne Stroustrup.

\item \textbf{Capacità di processo:} è un parametro che consente di valutare quanto un processo produttivo, riesca a soddisfare una specifica di produzione. Indica inoltre la riproducibilità del prodotto in uscita da un processo.

\item \textbf{Caso d'uso:} tecnica usata nei processi di ingegneria del software per effettuare in maniera esaustiva e non ambigua, la raccolta dei requisiti al fine di produrre software di qualità.
Consiste nel valutare ogni requisito, focalizzandosi sugli attori che interagiscono col sistema e valutandone le varie interazioni. Il documento dei casi d'uso, individua e descrive gli scenari elementari di utilizzo del sistema, da parte degli attori che si interfacciano con esso.

\item \textbf{Ciclo di Deming:} è un modello studiato per il miglioramento continuo della qualità in un'ottica a lungo raggio. Serve per promuovere una cultura della qualità che è tesa al miglioramento continuo dei processi e all'utilizzo ottimale delle risorse. Questo strumento parte dall'assunto che per il raggiungimento del massimo della qualità sia necessaria la costante interazione tra ricerca, progettazione, test, produzione e vendita. Per migliorare la qualità e soddisfare il cliente, le quattro fasi devono ruotare costantemente, tenendo come criterio principale la qualità.
\\La sequenza logica dei quattro punti ripetuti per un miglioramento continuo è la seguente:
\begin{itemize}
	\item \textbf{P:} Plan. Pianificazione;
	\item \textbf{D:} Do. Esecuzione del programma, dapprima in contesti circoscritti;
	\item \textbf{C:} Check. Test e controllo, studio e raccolta dei risultati e dei riscontri;
	\item \textbf{A:} Act. Azione per rendere definitivo e/o migliorare il processo.
\end{itemize}

\item\textbf{Cluster:} è un sottogruppo di una realtà complessa (la scomposizione dell'immagine).

\item\textbf{Cluster analysis:} in italiano \textit{analisi dei gruppi}, è un insieme di tecniche di analisi di dati, volte alla sezione e al raggruppamento di elementi omogenei in un insieme di dati. Per questo progetto si considerano tre principali algoritmi di clustering\glossario{}:
	\begin{itemize}
	\item \textbf{Fuzzy-c-means:} vedi Fuzzy-c-means\glossario{}.
	
	\item \textbf{Hierarchical:} vedi Hierarchical\glossario{}.

	\item \textbf{K-means:} vedi K-means\glossario{}.
	\end{itemize}
\end{itemize}
\pagebreak

%%%%%%%%%%%%%%%%%%%%%%%%%%%%%%%%%%%%%%%%%%%%%%%%%%%%%%%
%DDDDDDDDDDDDDDDDDDDDDDDDDDDDDDDDDDDDDDDDDDDDDDDDDDDDDD
%%%%%%%%%%%%%%%%%%%%%%%%%%%%%%%%%%%%%%%%%%%%%%%%%%%%%%%
\Huge D
\rhead{D}
\normalsize
\begin{itemize}
\item\textbf{Dataset:} astrazione logica di un analisi di una serie di Subject\glossario{}. È composto da un gruppo di Subject\glossario{} e dal Protocol\glossario{} che li processa.

\item \textbf{Design pattern:} soluzione progettuale generale per risolvere un problema ricorrente. Non è una libreria o un componente di software riusabile, ma un modello da applicare per risolvere un problema, che può presentarsi durante la fase di progettazione e sviluppo del software.

\item\textbf{Diagramma di Gantt:} strumento di supporto per la gestione dei progetti e delle risorse umane ad esso assegnate. \`E composto da un asse orizzontale in cui si visualizza l'incremento del tempo e da un asse verticale in cui sono elencate le attività da svolgere. Permette inoltre di definire delle dipendenze tra le attività.

\item \textbf{Diagramma di PERT:} acronimo di Program Evaluation and Review Technique, è un diagramma reticolare, che descrive la sequenza cronologica delle azioni pianificate per il completamento di un progetto nel suo complesso. Esso rappresenta graficamente il piano d’azione; solitamente si tratta di un sistema relativamente complesso, che richiede costi più elevati per la sua realizzazione e continuo aggiornamento. Inoltre pianifica le azioni, considerando le risorse a disposizione illimitate.

\item\textbf{Driver:} è una porzione di codice che si occupa di chiamare altri metodi, con l'unica finalità di verificarli tramite dei test.

\item \textbf{Dropbox:} servizio gratuito che permette di portare file e documenti ovunque e condividerli facilmente con gli altri. Offre dunque un servizio di archiviazione file e sincronizzazione automatica di file tramite web, disponibile per sistemi operativi e mobile.
\end{itemize}
\pagebreak

%\Huge E
%\rhead{E}
%\normalsize
%\begin{itemize}

%\end{itemize}