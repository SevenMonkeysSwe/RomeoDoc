%%%%%%%%%%%%%%%%%%%%%%%%%%%%%%%%%%%%%%%%%%%%%%%%%%%%%%%
%FFFFFFFFFFFFFFFFFFFFFFFFFFFFFFFFFFFFFFFFFFFFFFFFFFFFFF
%%%%%%%%%%%%%%%%%%%%%%%%%%%%%%%%%%%%%%%%%%%%%%%%%%%%%%%
\Huge F
\rhead{F}
\normalsize
\begin{itemize}
\item\textbf{Feature:} informazione estratta da un immagine non presente originariamente. Può essere:
\begin{itemize}
\item\textbf{No time dependent:} vedi No time dependent\glossario{}.
\item\textbf{Time dependent:} vedi Time dependent\glossario{}.
\end{itemize}

\item\textbf{Feature extractor:} funzione che calcola una feature\glossario{} d'interesse su un immagine;

\item\textbf{fMRI (\textit{functional Magnetic Resonance Imaging}):} è una tecnica di imaging biomedico\glossario{} che utilizza la risonanza magnetica, per valutare le funzionalità di un organo o di un apparato.

\item\textbf{Framework:} è una struttura logica di supporto allo sviluppo di software. Alla base ci sono una serie di librerie di codice, utilizzabili con uno o più linguaggi di programmazione. Spesso è corredato da una serie di strumenti per lo sviluppo, quali IDE\glossario{}, debuggers ecc\dots .

\item\textbf{Fuzzy-c-means:} algoritmo che divide un insieme di N elementi in K cluster. Ciascun dato può essere assegnato a più di un cluster; l'algoritmo ritorna la probabilità che un dato appartenga a un certo cluster; 
\end{itemize}
\pagebreak

%%%%%%%%%%%%%%%%%%%%%%%%%%%%%%%%%%%%%%%%%%%%%%%%%%%%%%%
%GGGGGGGGGGGGGGGGGGGGGGGGGGGGGGGGGGGGGGGGGGGGGGGGGGGGGG
%%%%%%%%%%%%%%%%%%%%%%%%%%%%%%%%%%%%%%%%%%%%%%%%%%%%%%%
\Huge G
\rhead{G}
\normalsize
\begin{itemize}
\item\textbf{Gantt:} vedi Diagramma di Gantt\glossario{}.

\item\textbf{GanttProject:} software open source, sviluppato in Java, utilizzato per la gestione delle risorse nei progetti. Permette di creare Diagrammi di Gantt\glossario{}.

\item\textbf{Gerarchico bottom-up:} vedi Hierarchical\glossario{}.

\item\textbf{Gerarchico agglomerativo:} vedi Hierarchical\glossario{}.

\item \textbf{Git:} sistema software di controllo di versione distribuito, creato da Linus Torvalds nel 2005.

\item \textbf{GitHub:} servizio web di hosting per lo sviluppo di progetti software, che usa il sistema di controllo di versione Git\glossario{}. GitHub\glossario{} offre la possibilità di gestire repository\glossario{} privati a pagamento o pubblici, molto utilizzati per lo sviluppo di progetti open source.

\item \textbf{Google Calendar:} servizio web fornito da Google, che permette di utilizzare dei calendari personalizzati e condivisibili con altri utenti.

\item \textbf{Google Docs:} servizio web fornito da Google che permette di produrre documenti in modo collaborativo e di salvare al suo interno documenti di testo e fogli di calcolo.

\item \textbf{Google Drive:} servizio web fornito da Google, che permette la condivisione e la modifica concorrente di documenti online.

\item \textbf{Google Hangouts:} servizio web offerto da Google che mette a disposizione una chat per poter comunicare con gli utenti e, grazie all'installazione di un plugin, permette di fare videochiamate tra due o più persone.

\item \textbf{GUI:} sta per interfaccia grafica utente, è il mezzo con cui l'utente interagisce con il sistema.
\end{itemize}
\pagebreak

%%%%%%%%%%%%%%%%%%%%%%%%%%%%%%%%%%%%%%%%%%%%%%%%%%%%%%%
%HHHHHHHHHHHHHHHHHHHHHHHHHHHHHHHHHHHHHHHHHHHHHHHHHHHHHH
%%%%%%%%%%%%%%%%%%%%%%%%%%%%%%%%%%%%%%%%%%%%%%%%%%%%%%%
\Huge H
\rhead{H}
\normalsize
\begin{itemize}
\item\textbf{Hierarchical:} (detto anche Gerarchico agglomerativo\glossario{} o Gerarchico bottom-up\glossario{}) è un algoritmo che raggruppa i dati in un albero di cluster\glossario{} e lavora basandosi sulla matrice di dissimilarità. L'algoritmo è di tipo «hard» (ogni dato assegnato ad uno ed un solo cluster);
\end{itemize}
\pagebreak

%%%%%%%%%%%%%%%%%%%%%%%%%%%%%%%%%%%%%%%%%%%%%%%%%%%%%%%
%IIIIIIIIIIIIIIIIIIIIIIIIIIIIIIIIIIIIIIIIIIIIIIIIIIIIII
%%%%%%%%%%%%%%%%%%%%%%%%%%%%%%%%%%%%%%%%%%%%%%%%%%%%%%%
\Huge I
\rhead{I}
\normalsize
\begin{itemize}
\item\textbf{IDE (\textit{Integrated Development Environment}):} software che, in fase di sviluppo, aiuta i programmatori nella scrittura del codice sorgente. Normalmente è composto da più componenti, quali:
\begin{itemize}
\item\textbf{Editor:} per comporre il codice. Spesso, questi editor applicano automaticamente un markup stilistico al codice sorgente, aumentandone la comprensione;
\item\textbf{Compilatore e/o Interprete:} possono essere integrati direttamente nel software, per il linguaggio principale su cui opera;
\item\textbf{Strumento di costruzione automatico:} può essere presente un tool che permette di costruire automaticamente il \textit{Makefile};
\item\textbf{Debugger:} solitamente è presente uno strumento che aiuta ad individuare gli errori presenti nel codice.
\end{itemize}

\item\textbf{IEC (\textit{International Electrotechnical Commission}):} è un'organizzazione internazionale per la definizione di standard in materia di elettricità, elettronica e tecnologie correlate. Molti dei suoi standard sono definiti in collaborazione con l'ISO\glossario{}. Questa commissione è formata da rappresentanti di enti di standardizzazione nazionali riconosciuti.

\item\textbf{Imaging biomedico:} è un generico processo, attraverso il quale è possibile osservare un area dell'organismo umano, non visibile dall'esterno.

\item\textbf{ISO (\textit{International Organization for Standardization}):} è la più importante organizzazione a livello mondiale per la definizione di norme tecniche. Fondata il 23 febbraio 1947, ha il suo quartier generale a Ginevra in Svizzera. Membri dell'ISO sono gli organismi nazionali di standardizzazione di 162 Paesi del mondo. L'ISO coopera strettamente con l'IEC\glossario{}, responsabile per la standardizzazione degli equipaggiamenti elettrici.

\item \textbf{ITK (\textit{Insight Segmentation and Registration Toolkit}):} libreria che offre svariati strumenti per l'analisi di immagini. In particolare, consente di immagazzinare e operare anche su immagini biomediche.
\end{itemize}
\pagebreak

%%%%%%%%%%%%%%%%%%%%%%%%%%%%%%%%%%%%%%%%%%%%%%%%%%%%%%%
%JJJJJJJJJJJJJJJJJJJJJJJJJJJJJJJJJJJJJJJJJJJJJJJJJJJJJJ
%%%%%%%%%%%%%%%%%%%%%%%%%%%%%%%%%%%%%%%%%%%%%%%%%%%%%%%
\Huge J
\rhead{J}
\normalsize
\begin{itemize}
\item\textbf{JPEG (\textit{Joint Photographic Experts Group}):} indica un formato dati per la rappresentazione di immagini bitmap\g{}, le principali estensioni sono: \verb!JPG!, \verb!JPEG! e \verb!JPE!. Tale formato è il più usato nel contesto della fotografia digitale in quanto è un formato di compressione con bassa perdita di qualità.

\item\textbf{JPG:} è una delle estensioni usate dal formato \verb!JPEG!\glossario{}.
\end{itemize}
\pagebreak