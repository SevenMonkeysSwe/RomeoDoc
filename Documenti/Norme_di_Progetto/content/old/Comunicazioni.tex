\section{Comunicazioni}
\label{comunicazioni}

\subsection{Comunicazioni esterne}
\label{cesterne}
Per le comunicazioni verso l'esterno è stato creato un indirizzo di posta elettronica apposito: \email.
\\Tale indirizzo dovrà essere l'unico servizio utilizzabile per le comunicazioni verso l'esterno.
\\Sarà solo il \projectManager{} a utilizzare l'indirizzo di posta
 per conto del gruppo \authorName{} intrattenendo le corrispondenze con i proponenti e i committenti.
\\Eventualmente il \projectManager{} provvederà ad inoltrare le conversazioni a tutti i membri del gruppo tramite la mailing list\glossario{}, ma solo se ritiene che sia necessario.

\subsection{Comunicazioni interne}
\label{cinterne}
Le comunicazioni interne verranno eseguite tramite la mailing list\glossario{}: \email.
\\Quando un membro del gruppo vuole inviare un'email a tutti i componenti, deve inviare il messaggio dalla sua email personale verso l'indirizzo \email{}. Un inoltro automatico provvederà a trasmettere l'email agli indirizzi personali dei componenti del gruppo presenti nella mailing list\glossario{}, tranne che al membro che ha inviato il messaggio. In questo modo tutti i componenti saranno sempre al corrente di tutti gli incontri e impegni del gruppo.
\\Al fine di facilitare le comunicazioni tra i vari membri del gruppo,
 viene utilizzato Google Hangout\glossario{} come servizio di messaggistica istantanea e per le video chiamate.
\\È necessario redigere un verbale\footnote{Vedi sezione \ref{verbali} per maggiori dettagli.} nel caso in cui siano state prese decisioni o siano emersi dettagli inerenti allo sviluppo del progetto.

\subsection{Composizione email}
\label{composizione}
In questo paragrafo è descritta la forma che deve avere una email sia per una comunicazione interna che esterna.

\subsubsection{Mittente}
\label{mittente}
Il mittente della email potrà cambiare a seconda del tipo di comunicazione svolta:
\begin{itemize}
\item\textbf{Esterna:} l'unico indirizzo utilizzabile per comunicare verso l'esterno dovrà essere \textit{necessariamente} l'indirizzo \email, il quale sarà usato \textit{esclusivamente} dal \projectManager{};
\item \textbf{Interna:} in questo caso andrà messo l'indirizzo personale di chi scrive.
\end{itemize}

\subsubsection{Destinatario}
\label{destinatario}
Il destinatario della e-mail cambierà a seconda che si tratti di una comunicazione interna o esterna:
\begin{itemize}
\item \textbf{Esterna:} l'indirizzo del destinatario potrà variare a seconda che si voglia comunicare con il Prof. Tullio Vardanega, il Prof. Riccardo Cardin o con i proponenti del progetto;
\item \textbf{Interna:} l'unico indirizzo utilizzabile è \email.
\end{itemize}
Sono ammesse alcune eccezioni:
\begin{itemize}
\item \textbf{Proposta all'\administrator{}:} nel caso in cui un membro del gruppo voglia contattare l'\administrator{}
 per richiedere cambiamenti alle norme, il membro dovrà contattarlo al suo indirizzo di posta personale\footnote{Sarà possibile trovare i vari recapiti e-mail personali nel documento \lq\lq{}contatti\rq\rq{} condiviso su Google Drive.};
\item \textbf{Proposta al \projectManager{}:} nel caso in cui un membro del gruppo voglia richiedere una riunione, dovrà contattarlo al suo indirizzo di posta personale\footnote{Vedi nota precedente.};
\item \textbf{Comunicazione ristretta tra alcuni membri del team:} in alcuni casi i membri del team potrebbero avere la necessità di comunicare tra di loro e utilizzeranno i loro indirizzi personali\footnote{Vedi due note precedenti.}.
\end{itemize}

\subsubsection{Oggetto}
\label{oggetto}
L'oggetto deve essere chiaro, esaustivo e possibilmente univoco, in modo da riconoscerlo da quelli precedenti.
\\Nel caso si debba inviare un messaggio alla mailing list\glossario{}, vi è l'obbligo di aggiungere \lq\lq\textbf{CI:}\rq\rq{} all'inizio dell'oggetto. 

\subsubsection{Corpo}
\label{corpo}
Il corpo di un messaggio dovrà avere tutti gli elementi e le informazioni che permettano a tutti i destinatari di capire correttamente l'argomento trattato.
\\Se alcune parti del massaggio si riferiscono a particolari membri del gruppo o a certi ruoli di progetto si dovrà usare la seguente sintassi: \lq\lq{}\textbf{@Cognome Nome}\rq\rq{} o 
\lq\lq{} \textbf{@Nome Ruolo}\rq\rq{} per riferirsi ad essi.
Nel caso in cui il corpo del messaggio abbia più di trenta righe, è preferibile scrivere un corpo riassuntivo ed allegare un file \verb!PDF!\glossario{} che scenda più nel dettaglio.
\\Alla fine del corpo il mittente dovrà sempre firmarsi col suo cognome, nome e ruolo.

\subsubsection{Allegati}
\label{allegati}
Viene consentito di allegare dei file al messaggio, preferibilmente in formato \verb!PDF!\glossario{}, i quali non dovranno superare i 20MB.
\\Essi potranno essere utilizzati ad esempio per allegare il verbale di un incontro.