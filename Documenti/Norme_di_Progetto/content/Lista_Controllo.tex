\section{Lista di controllo}
\label{lista_di_controllo}
Nella verifica dei documenti, si sono rilevati, più frequentemente, errori dovuti a:
\begin{itemize}
\item \textbf{Non aderenza delle Norme di Progetto:} alcuni errori ricorrenti sono dovuti alla non piena conoscenza, da parte dei redattori, delle norme riguardanti la stesura dei documenti (vedi sezione \ref{norme tipografiche}) in particolare:
	\begin{itemize}
	\item Inserito il \lq\lq{.}\rq\rq{} oppure nessun simbolo di punteggiatura al posto del \lq\lq{;}\rq\rq{} negli elenchi puntati al termine di ogni voce interna;
	\item Errato utilizzo del markup nei riferimenti delle estensioni ai file;
	\item Inserimento lettera maiuscola su titoli paragrafi ove non necessario;
	\item Mancanza, nei termini di glossario, della marcatura \lq\lq{G}\rq\rq{} in pedice;
	\item Non utilizzato il corsivo per il riferimento ad altri documenti;
	\item Mancato utilizzo dei comandi creati per alcuni termini di uso ricorrente.
	\end{itemize}
\item \textbf{Errori ortografici:}
	\begin{itemize}
	\item Eccessivo utilizzo del punto e virgola nei periodi. \`E preferibile inserire il punto;
	\item Dovuti a distrazioni e/o errori di battitura;
	\item Punteggiatura sbagliata o mancante.
	\end{itemize}
\item \textbf{Errori riguardanti la lingua italiana:}
	\begin{itemize}
	\item Passaggio da un tempo verbale ad un altro nello stesso periodo;
	\item Utilizzo di parole il cui significato attribuito dal redattore, non è conforme a quello dato dalla lingua italiana;
	\item Inserimento della virgola tra soggetto e verbo. Rende poco comprensibile il periodo.
	\end{itemize}
\item \textbf{Errori riguardanti la stesura dei diagrammi UML:}
	\begin{itemize}
	\item Nei diagrammi delle classi, le classi e i metodi astratti non vengono riportati in corsivo;
	\item Nei diagrammi delle classi non viene utilizzata l'esatta relazione di implementazione di un'interfaccia;
	\item Nei diagrammi delle classi è stata utilizzata la relazione d'uso invece della relazione di creazione;
	\item Mancata coerenza tra nomi presenti nei documenti e nomi nei diagrammi;
	\item Nei digrammi delle classi è stata utilizzata erroneamente la relazione di aggregazione invece che quella di composizione;
	\item Nei digrammi di sequenza mancato inserimento del messaggio di ritorno.
	\end{itemize}
\end{itemize}
