\section{Incontri}
\label{incontri}
Sarà il \projectManager{} a fissare gli incontri, sia interni che esterni.
\\Per ogni incontro dovrà essere specificata data, ora, luogo, ordine del giorno e motivi della riunione; tali informazioni dovranno essere rese disponibili con almeno quattro giorni di anticipo.
\\Successivamente alla decisione di una nuova data il \projectManager{} provvederà alla creazione di un evento sul calendario condiviso su Google Calendar\glossario{}.
\subsection{Incontri esterni}
\label{iesterni}
Sarà il \projectManager{} a fissare gli incontri con i proponenti e/o committenti utilizzando la casella di posta creata appositamente\footnote{Si veda il paragrafo \textit{Strumenti di comunicazione} della sezione \ref{definizioneinfrastruttura}}.
Il \projectManager{} prima di prendere alcun accordo con le parti esterne dovrà contattare i vari componenti del gruppo, per sentire se almeno cinque membri concordano.
\\In caso positivo il \projectManager{} provvederà a contattare i proponenti e/o committenti per fissare la data dell'incontro.
\\Sarà compito del \projectManager{} redigere il verbale dell'incontro avvenuto.
\subsection{Incontri interni}
\label{iinterni}
Sarà il \projectManager{} a fissare gli incontri interni, contattando tutti i membri del gruppo. Gli incontri interni dovranno avere una frequenza almeno quindicinale.
Ogni componente del gruppo è tenuto a leggere regolarmente la posta elettronica personale e rispondere ad eventuali richieste di un incontro interno.
Nel caso in cui i membri disponibili a partecipare alla riunione in una certa data siano meno di quattro, questa verrà posticipata o anticipata, in modo che siano presenti un numero adeguato di persone.
Inoltre, è possibile e auspicabile che siano necessarie riunioni tra specifici membri del gruppo, ad esempio: in fase di analisi può essere utile che solo gli \emph{Analisti} si incontrino tra di loro, senza il resto del gruppo.
I restanti componenti del gruppo saranno comunque informati sui contenuti e le decisioni prese tramite invio di una e-mail alla mailing list\glossario{} o alla pubblicazione di un verbale sul repository\glossario{} nel caso siano state prese decisioni importanti.
\subsection{Richieste di incontri}
\label{rincontri}
Qualora ve ne sia la necessità, ogni membro del gruppo potrà richiedere un incontro sia interno che esterno, contattando personalmente il \projectManager{} ed esponendo i motivi della richiesta.
\\Il \projectManager{} avrà potere di scegliere se accettare o rifiutare la richiesta.