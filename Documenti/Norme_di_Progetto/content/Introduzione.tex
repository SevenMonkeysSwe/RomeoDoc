\section{Introduzione}
\label{introduzione}
\subsection{Scopo del documento}
\label{scopodocumento}
Questo documento ha lo scopo di fissare le norme che tutti i membri del gruppo \authorName{}
dovranno rispettare durante lo svolgimento del progetto \project.
\\Tutti i membri sono tenuti a leggere il documento e a seguire \textit{rigorosamente} le norme ivi descritte, al fine di garantire un lavoro efficiente, efficace e per avere un'uniformità nei documenti prodotti.
\\Qualora ve ne sia la necessità, ogni membro del gruppo potrà contattare l'\administrator{} per suggerire l'aggiunta di nuove norme, oppure cambiamenti alle norme già esistenti.\\ 
L'\administrator{}, dopo essersi consultato con gli altri membri del gruppo, avrà la facoltà di accettare o rifiutare i suggerimenti proposti.
\\Le norme esposte nel documento riguarderanno tutti gli aspetti che sono inerenti allo sviluppo del prodotto software.
\\Il documento, in particolare, pone l'accento sui seguenti punti:
\begin{itemize}
\item Le interazioni tra i vari membri del gruppo e verso l'esterno;
\item Le modalità di accesso al repository\glossario{};
\item La stesura dei documenti e le varie convenzioni di scrittura utilizzate;
\item Le norme utilizzate nella scrittura del codice;
\item La definizione dell'ambiente di lavoro.
\end{itemize}

\subsection{Glossario}
\label{glossario}
\glossIntro{}

\subsection{Riferimenti}
\label{riferimenti}

\subsubsection{Normativi}
\label{rnormativi}
\begin{itemize}
\item \textbf{Qt Coding Convenctions:} \url{http://qt-project.org/wiki/Coding-Conventions};
\item \textbf{Qt Coding Style:} \url{http://qt-project.org/wiki/Qt_Coding_Style}.
\end{itemize}

\subsubsection{Informativi}
\label{rinformativi}
\begin{itemize}
\item \PdP{}
\end{itemize}