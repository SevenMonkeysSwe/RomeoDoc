\documentclass[11pt,a4paper]{article}
	\usepackage{fixltx2e}
	% Packages per la formattazione
	\usepackage{subcaption} %package per le sottoimmagini
	\usepackage{titlesec} % %package per le sezioni a quattro indici
	\usepackage[left=3.5cm,right=3.5cm,top=3.1cm,bottom=2.6cm]{geometry}
	\usepackage[english]{babel}
	\usepackage[utf8]{inputenc}
	\usepackage{lastpage}
	\usepackage[T1]{fontenc}
	\usepackage{geometry}
	\usepackage{graphicx}%package per la gestione delle immagini
	\geometry{a4paper}
	\usepackage{fancyhdr} %package di style per  l/c/rhead l/rfoot
	\usepackage{tocloft} %package per i punti nella tableofcontents
	\usepackage{xr} %package per i riferimenti a label di file esterni
	\usepackage{longtable}
	\pagestyle{fancy}
	\usepackage{xcolor} % package per i colori
	\usepackage{color} % sempre per i colori
	\usepackage{verbatim} %per scrivere codice e inserire commenti di più righe
	\definecolor{linkcolor}{cmyk}{1,.60,0,.40}
	\usepackage{hyperref}
		\hypersetup{
			colorlinks=true,
			linkcolor=black,
			urlcolor=linkcolor
		}		
	\usepackage{tabularx} %package meno complesso per la gestione delle tabelle
%	\usepackage{svg} % package necessario per includere immagini .svg
	\usepackage{pdfpages}
	\usepackage{tikz} % package per disegnare figure
	\usepackage{eurosym} %package per il simbolo dell'euro
	\usepackage[font=small,labelfont=bf]{caption} %pakage per ridefinire font caption
	\usepackage{multirow} %package per unire 2 righe
	\usepackage{pdflscape} %package per mettere il foglio in orizzontale (utile in caso di figure che si estendono in orizzontale)
	\usepackage{titlesec}