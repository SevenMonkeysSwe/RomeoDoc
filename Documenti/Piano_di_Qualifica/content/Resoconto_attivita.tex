\externaldocument{visione_generale}
\section{Resoconto delle attività di verifica}
\label{resoconto_attivita}
Qui verranno riportati gli esiti delle attività di verifica effettuate durante le varie fasi del progetto.

\subsection{Revisione dei Requisiti}
\label{RR}
%\subsubsection{Processi}
%\label{proRR}
%Vengono di seguito riportati i valori degli indici SV\g{} e BV\g{} descritti nella sezione \ref{metriche_processi} per i processi della Fase A (FA).
%\begin{table}[!h]
%\begin{center}
%		\begin{tabular}{|p{5cm}|c|c|}
%			\hline
%			\textbf{Processo} &
%			\textbf{SV\g{}} &
%			\textbf{BV\g{}}\\ \hline
%			\textit{ } &  & \\ 
%			\textit{} & & \\ 
%			\textit{} & & \\ 
%			\textit{} & & \\ 
%			\textit{} & & \\ 
%			\textit{} & & \\ 
%			\hline
%		\end{tabular}
%		\caption{Esiti verifica processi, Fase A}
%\end{center}
%\end{table}
%Complessivamente la Fase A ha:
%\begin{itemize}
%\item \textbf{SV\g{}} = ;
%\item \textbf{BV\g{}} = .
%\end{itemize}
%Si può dunque dedurre che:
%\begin{itemize}
%\item ;
%\item ;
%\item.
%\end{itemize}

\subsubsection{Documenti}
\label{docRR}
A seguito delle attività di verifica effettuate sulla documentazione, sono state ricavate le seguenti misurazioni:
\begin{table}[!h]
\begin{center}
		\begin{tabular}{|p{5cm}|c|c|}
			\hline
			\textbf{Documento} &
			\textbf{Indice Gulpease} &
			\textbf{Esito}\\ \hline
			\textit{Piano di Qualifica v1.0.0} & 49 &  \textit{\color{green}Superato} \\ 
			\textit{Piano di Progetto v1.0.0} & 48 & \textit{\color{green}Superato} \\ 
			\textit{Analisi dei Requisiti v1.0.0} & 56 & \textit{\color{green}Superato} \\ 
			\textit{Norme di Progetto v1.0.0} & 54 & \textit{\color{green}Superato} \\ 
			\textit{Studio di Fattibilità v1.0.0} & 52 & \textit{\color{green}Superato} \\ 
			\textit{Glossario v1.0.0} & 55 & \textit{\color{green}Superato} \\ 
			\hline
		\end{tabular}
		\caption{Indice Gulpease dei documenti presentati in Revisione dei Requisiti}
\end{center}
\end{table}
\\Tutti i documenti verificati, soddisfano i range di accettazione specificati nella sezione \ref{metriche_documentazione}. Alcuni di essi però, non rientrano nel range ottimale. Questo implica quindi, una futura ricerca e modifica di frasi troppo complesse, al fine di migliorare la lettura dei documenti e rientrare nel range ottimale.
\pagebreak
\subsection{Revisione di Progettazione}
\label{RP}
%\subsubsection{Processi}
%\label{proRP}
%Vengono di seguito riportati i valori degli indici SV\g{} e BV\g{} descritti nella sezione \ref{metriche_processi} per i processi della Fase B (FB) e Fase C (FC).
%\begin{table}[!h]
%\begin{center}
%		\begin{tabular}{|p{5cm}|c|c|}
%			\hline
%			\textbf{Processo} &
%			\textbf{SV\g{}} &
%			\textbf{BV\g{}}\\ \hline
%			\textit{ } &  & \\ 
%			\textit{} & & \\ 
%			\textit{} & & \\ 
%			\textit{} & & \\ 
%			\textit{} & & \\ 
%			\textit{} & & \\ 
%			\hline
%		\end{tabular}
%		\caption{Esiti verifica processi, Fase B e Fase C}
%\end{center}
%\end{table}
%Complessivamente la Fase B e Fase C hanno:
%\begin{itemize}
%\item \textbf{SV\g{}} = ;
%\item \textbf{BV\g{}} = .
%\end{itemize}
%Si può dunque dedurre che:
%\begin{itemize}
%\item ;
%\item ;
%\item .
%\end{itemize}
\subsubsection{Documenti}
\label{docRP}
A seguito delle attività di verifica effettuate sulla documentazione, sono state ricavate le seguenti misurazioni:
\begin{table}[!h]
\begin{center}
		\begin{tabular}{|p{5cm}|c|c|}
			\hline
			\textbf{Documento} &
			\textbf{Indice Gulpease} &
			\textbf{Esito}\\ \hline
			\textit{Piano di Qualifica v2.0.0} & 55 &  \textit{\color{green}Superato} \\ 
			\textit{Piano di Progetto v2.0.0} & 53 & \textit{\color{green}Superato} \\ 
			\textit{Analisi dei Requisiti v2.0.0} & 65 & \textit{\color{green}Superato} \\ 
			\textit{Norme di Progetto v2.0.0} & 57 & \textit{\color{green}Superato} \\ 
			\textit{Specifica Tecnica v1.0.0} & 49 & \textit{\color{green}Superato} \\ 
			\textit{Glossario v2.0.0} & 55 & \textit{\color{green}Superato} \\ 
			\hline
		\end{tabular}
		\caption{Indice Gulpease dei documenti presentati in Revisione di Progettazione}
\end{center}
\end{table}
\\Tutti i documenti verificati, soddisfano i range di accettazione specificati nella sezione \ref{metriche_documentazione}. Alcuni di essi però, non rientrano nel range ottimale. Questo implica quindi, una futura ricerca e modifica di frasi troppo complesse, al fine di migliorare la lettura dei documenti e rientrare nel range ottimale.

\subsection{Revisione di Qualifica}
\label{RQ}
%\subsubsection{Processi}
%\label{proRQ}
%Vengono di seguito riportati i valori degli indici SV\g{} e BV\g{} descritti nella sezione \ref{metriche_processi} per i processi della Fase D (FD).
%\begin{table}[!h]
%\begin{center}
%		\begin{tabular}{|p{5cm}|c|c|}
%			\hline
%			\textbf{Processo} &
%			\textbf{SV\g{}} &
%			\textbf{BV\g{}}\\ \hline
%			\textit{ } &  & \\ 
%			\textit{} & & \\ 
%			\textit{} & & \\ 
%			\textit{} & & \\ 
%			\textit{} & & \\ 
%			\textit{} & & \\ 
%			\hline
%		\end{tabular}
%		\caption{Esiti verifica processi, Fase D}
%\end{center}
%\end{table}
%Complessivamente la Fase D ha:
%\begin{itemize}
%\item \textbf{SV\g{}} = ;
%\item \textbf{BV\g{}} = .
%\end{itemize}
%Si può dunque dedurre che:
%\begin{itemize}
%\item ;
%\item ;
%\item .
%\end{itemize}

\subsubsection{Documenti}
\label{docRQ}
A seguito delle attività di verifica effettuate sulla documentazione, sono state ricavate le seguenti misurazioni:
\begin{table}[!h]
\begin{center}
		\begin{tabular}{|p{5cm}|c|c|}
			\hline
			\textbf{Documento} &
			\textbf{Indice Gulpease} &
			\textbf{Esito}\\ \hline
			\textit{Piano di Qualifica v3.0.0} & 55 &  \textit{\color{green}Superato} \\ 
			\textit{Piano di Progetto v3.0.0} & 52 & \textit{\color{green}Superato} \\ 
			\textit{Analisi dei Requisiti v3.0.0} & 65 & \textit{\color{green}Superato} \\ 
			\textit{Norme di Progetto v3.0.0} & 55 & \textit{\color{green}Superato} \\ 
			\textit{Specifica Tecnica v2.0.0} & 51 & \textit{\color{green}Superato} \\ 
			\textit{Glossario v3.0.0} & 55 & \textit{\color{green}Superato} \\ 
			\textit{Manuale Utente v1.0.0}& 59 &\textit{\color{green}Superato}\\
			\textit{Definizione di Prodotto v1.0.0}& 50 &\textit{\color{green}Superato}\\
			\hline
		\end{tabular}
		\caption{Indice Gulpease dei documenti presentati in Revisione di Qualifica}
\end{center}
\end{table}
\\Tutti i documenti verificati.

\subsubsection{Progettazione}
Viene di seguito riportata una tabella, nella quale è indicato il livello di accoppiamento afferente ed efferente, individuato per ogni componente rilevato durante la progettazione architetturale.\\

\begin{tabular}{|p{7cm}|c|c|}
			\hline
			\textbf{Componente} &
			\textbf{Afferente} &
			\textbf{Efferente}\\
			\hline
			Romeo::Model & 0 & 0\\ 
			\hline
			Romeo::Model::Core & 5 & 2\\ 
			\hline
			Romeo::Model::Core::Algorithm & 2 & 0\\ 
			\hline
			Romeo::Model::Core::Feature & 2 & 0\\ 
			\hline
			Romeo::Model::Util & 0 & 0\\ 
			\hline
			Romeo::Model::Util::Log & 1 & 2\\ 
			\hline
			Romeo::Model::Util::ReaderModel & 1 & 0\\
			\hline
			Romeo::Model::Util::ExporterModel & 1 & 0\\
			\hline
			Romeo::Model::Util::DAO & 1 & 3\\
			\hline
			Romeo::Model::QtModel & 1 & 0\\
			\hline
			Romeo::Model::Help & 2 & 0\\
			\hline
			Romeo::View & 0 & 0\\
			\hline
			Romeo::View::Dialog & 1 & 1\\
			\hline
			Romeo::View::Component & 1 & 1\\
			\hline
			Romeo::View::Window & 1 & 1\\
			\hline
			Romeo::Controller & 0 & 2\\
			\hline
		\end{tabular}



\subsubsection{Codice}
\label{codeRQ}
In questa sezione vengono riportati i risultati dei test di analisi statica e analisi dinamica effettuati sul codice finora prodotto.
Per ogni test effettuato vengono indicati i valori medi ed i valori massimi ottenuti, giustificando qualora ci fossero valori non rientranti nel range di accettazione.
\paragraph{Complessità ciclomatica}
\begin{itemize}
\item{\textbf{Media:}} 4
\item{\textbf{Massimo:}} 10
\end{itemize}
Tutti i moduli rientrano nel range di accettazione.

\paragraph{Parametri per metodo}
\begin{itemize}
\item{\textbf{Media:}} 2.5
\item{\textbf{Massimo:}} 5
\end{itemize}
Tutti i moduli rientrano nel range di accettazione.

\paragraph{Attributi per classe}
\begin{itemize}
\item{\textbf{Media:}} 6.5
\item{\textbf{Massimo:}} 10
\end{itemize}
In questo caso alcuni moduli del package view non rientrano nel range di accettazione, questo è dovuto al fatto che le classi view sono composte da un gran numero di elementi per offrire tutte le funzionalità individuate in analisi dei requisiti.

\paragraph{Linee di codice per linee di commento}
\begin{itemize}
\item{\textbf{Media:}} 1.5
\end{itemize}
Tutti i moduli rientrano nel range di accettazione.

\paragraph{Logical SLOC}
\begin{itemize}
\item{\textbf{Media:}} 30
\item{\textbf{Massimo:}} 60
\end{itemize}
In questo caso alcuni moduli, soprattutto del package view, non rientrano nel range di accettazione, questo è dovuto a causa dell'elevata quantità di attributi che questi possiedono, e che vengono utilizzati per fornire le funzionalità individuate in analisi dei requisiti.

\paragraph{Livello di annidamento}
\begin{itemize}
\item{\textbf{Media:}} 2
\item{\textbf{Massimo:}} 5
\end{itemize}
Tutti i moduli rientrano nel range di accettazione.


\paragraph{Copertura del codice}
\begin{itemize}
\item{\textbf{Media:}} 65%
\end{itemize}
La percentuale di copertura del codice è stata calcolata tenendo conto del numero effettivo di linee di codice testate con i test di unità. Risultano quindi esclusi dal calcolo tutti i metodi \emph{getter} e \emph{setter}, oltre ai metodi virtuali puri, in quanto, data la loro semplicità, non sono stati considerati meritevoli di test.


\pagebreak

%%%%%%%%%%%%%%%%%%%%%%%%% RA %%%%%%%%%%%%%%%%%%%%%
\subsection{Revisione di Accettazione}
\label{RA}
\subsubsection{Documenti}
\label{docRA}
A seguito delle attività di verifica effettuate sulla documentazione, sono state ricavate le seguenti misurazioni:
\begin{table}[!h]
\begin{center}
		\begin{tabular}{|p{5cm}|c|c|}
			\hline
			\textbf{Documento} &
			\textbf{Indice Gulpease} &
			\textbf{Esito}\\ \hline
			\textit{Piano di Qualifica v4.0.0} & 55 &  \textit{\color{green}Superato} \\ 
			\textit{Piano di Progetto v4.0.0} & 56 & \textit{\color{green}Superato} \\ 
			\textit{Analisi dei Requisiti v3.0.0} & 65 & \textit{\color{green}Superato} \\ 
			\textit{Norme di Progetto v4.0.0} & 56 & \textit{\color{green}Superato} \\ 
			\textit{Specifica Tecnica v3.0.0} & 54 & \textit{\color{green}Superato} \\ 
			\textit{Glossario v4.0.0} & 55 & \textit{\color{green}Superato} \\ 
			\textit{Manuale Utente v2.0.0}& 58 &\textit{\color{green}Superato}\\
			\textit{Definizione di Prodotto v2.0.0}& 51 &\textit{\color{green}Superato}\\
			\hline
		\end{tabular}
		\caption{Indice Gulpease dei documenti presentati in Revisione di Accettazione}
\end{center}
\end{table}
\\Tutti i documenti verificati.

\subsubsection{Codice}
\label{codeRA}
In questa sezione vengono riportati i risultati dei test di analisi statica e analisi dinamica effettuati sul codice finale.
Per ogni test effettuato vengono indicati i valori medi ed i valori massimi ottenuti, giustificando qualora ci fossero valori non rientranti nel range di accettazione.
\paragraph{Complessità ciclomatica}
\begin{itemize}
\item{\textbf{Media:} 6} 
\item{\textbf{Massimo:} 9}
\end{itemize}
Tutti i moduli rientrano nel range di accettazione.

\paragraph{Parametri per metodo}
\begin{itemize}
\item{\textbf{Media:}5} 
\item{\textbf{Massimo:}6} 
\end{itemize}
Tutti i moduli rientrano nel range di accettazione.

\paragraph{Attributi per classe}
\begin{itemize}
\item{\textbf{Media:}5} 
\item{\textbf{Massimo:}18} 
\end{itemize}
In questo caso alcuni moduli del package view non rientrano nel range di accettazione, questo è dovuto al fatto che le classi view sono composte da un gran numero di elementi per offrire tutte le funzionalità individuate in analisi dei requisiti e per la posizione dei componenti grafici nella gui\g{}.

\paragraph{Linee di codice per linee di commento}
\begin{itemize}
\item{\textbf{Media:}}1.76365 
\end{itemize}

\paragraph{Logical SLOC}
\begin{itemize}
\item{\textbf{Media:}28} 
\item{\textbf{Massimo:}60}
\end{itemize}
In questo caso alcuni moduli, soprattutto del package view, non rientrano nel range di accettazione, questo è dovuto a causa dell'elevata quantità di attributi che questi possiedono, e che vengono utilizzati per fornire le funzionalità individuate in analisi dei requisiti.

\paragraph{Livello di annidamento}
\begin{itemize}
\item{\textbf{Media:}2}
\item{\textbf{Massimo:}6} 
\end{itemize}
%Tutti i moduli rientrano nel range di accettazione.


\paragraph{Copertura del codice}
\begin{itemize}
\item{\textbf{Media:} 87\%} 
\end{itemize}
La percentuale di copertura del codice è stata calcolata tenendo conto del numero effettivo di linee di codice testate con i test di unità. Risultano quindi esclusi dal calcolo tutti i metodi \emph{getter} e \emph{setter}, oltre ai metodi virtuali puri, in quanto, data la loro semplicità, non sono stati considerati meritevoli di test.


\subsubsection{Verifica della copertura dei requisiti}
La tabella seguente riporta la copertura dei requisiti; come da \emph{verbale 8 esterno del 04/06/2014} alcuni requisiti hanno subito variazione di importanza.
\begin{center}
\begin{longtable}{|c|c|c|c|}
\hline
\textbf{Categoria} & \textbf{Non Soddisfatti} & \textbf{Soddisfatti} & \textbf{Totali} \\
\hline
\textbf{Obbligatorio} &3 & 150 & 153 \\ 
\hline
\textbf{Opzionale} & 1 & 5 & 6 \\
\hline
\textbf{Desiderabile} & 22 & 4 & 26 \\
\hline
\textbf{Totali} & 26 & 159 & 185 \\
\hline
\caption{Riepilogo copertura dei requisiti}
\end{longtable}
\end{center}