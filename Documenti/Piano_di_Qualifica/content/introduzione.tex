\section{Introduzione}
\label{introduzione}

\subsection{Scopo del Documento}
\label{scopo_del_documento}
Il \textit{Piano di Qualifica} ha lo scopo di descrivere le strategie che il gruppo intende adottare, al fine di perseguire gli obbiettivi qualitativi che si intendono applicare al prodotto da sviluppare. Per soddisfare tali obbiettivi, è quindi necessario un processo di verifica costante sulle attività svolte, permettendo così di rilevare eventuali incongruenze e anomalie che si potrebbero riscontrare.

\subsection{Scopo del Prodotto}
\label{scopo_del_prodotto}
\scopoProd{}
\subsection{Glossario}
\label{glossario}
\glossIntro{}
\subsection{Riferimenti}
\label{riferimenti}

\subsubsection{Normativi}
\begin{itemize}
	\item \textbf{Norme di Progetto}: \NdP;
	\item \textbf{Capitolato d'appalto C3}: \project, Medical Image Cluster Analysis Tool
	\url{http://www.math.unipd.it/~tullio/IS-1/2013/Progetto/C3p.pdf}
	\item \textbf{Standard ISO\glossario /IEC\glossario 9126:} Software Engineering - Product Quality 
	\url{http://en.wikipedia.org/wiki/ISO/IEC_9126}
	\item \textbf{Standard ISO\glossario /IEC\glossario 15504 (SPICE):} Information Technology — Process assessment
	\begin{itemize}			
		\item \url{http://en.wikipedia.org/wiki/ISO/IEC_15504};
		\item \url{http://ww.ehealthinformation.ca/documents/1072.pdf}
	\end{itemize}
\end{itemize}
\subsubsection{Informativi}
\begin{itemize}
	\item \textbf{Piano di Progetto:} \PdP;
	\item \textbf{Slide del corsi di Ingegneria del Software modulo A:}\url{http://www.math.unipd.it/~tullio/IS-1/2013};
	\item \textbf{SWEBOK V3 (2004)}: Chapter 11 - Software Quality \url{http://www.computer.org/portal/web/swebok/html/ch11};

	\item \textbf{Indice Gulpease:} 
		\url{http://it.wikipedia.org/wiki/Indice_Gulpease}.
\end{itemize}

\pagebreak