
\subsection{Test di unità }
\label{testdiunita}

Di seguito vengono riportati i test di unità  previsti.

\newdimen\larghezza
\setlength{\larghezza}{6cm}
\newdimen\dimTipo
\setlength{\dimTipo}{5cm}
\begin{center}
\begin{longtable}{|c|c|c|c|}
\hline

\textbf{Test} & \textbf{Descrizione} & \textbf{Stato} & \textbf{Metodi} \\


\hline
TU1 & \parbox[t]{\larghezza}{ Si verifica che,passato il path di un immagine 2D di test valida,essa venga importata correttamente.Si verifica inoltre,che dato un immagine di test 2D di formato non valido,l'importazione fallisca } & Superato & \parbox[t]{\dimTipo} { ImageReader() \\ readFile2D() \\ } \\
\hline
TU2 & \parbox[t]{\larghezza}{ Si verifica che,passato il path di un immagine 3D di test valida,essa venga importata correttamente.Si verifica inoltre,che dato un immagine di test 3D di formato non valido,l'importazione fallisca } & Superato & \parbox[t]{\dimTipo} { ImageReader() \\ readFile3D() \\ } \\
\hline
TU3 & \parbox[t]{\larghezza}{ Si verifica che,passato il path di un video di test valido,esso venga importato correttamente.Si verifica inoltre,che dato un video di test di formato non valido,l'importazione fallisca } & Superato & \parbox[t]{\dimTipo} { VideoReader() \\ readFile2D() \\ readFile3D() \\ } \\
\hline
TU4 & \parbox[t]{\larghezza}{ Si verifica che,passato un path e un immagine 2D valida,essa venga esportata in maniera corretta e nella cartella indicata } & Superato & \parbox[t]{\dimTipo} { ImageExporter() \\ exportFile() \\ } \\
\hline
TU5 & \parbox[t]{\larghezza}{ Si verifica che,passato un path e un immagine 3D valida,essa venga esportata in maniera corretta e nella cartella indicata } & Superato & \parbox[t]{\dimTipo} { AnalyzeExporter() \\ exportFile() \\ NiftiExporter() \\ exportFile() \\ } \\
\hline
TU6 & \parbox[t]{\larghezza}{ Si verifica che dato un Subject ben formato,esso venga correttamente salvato nel database.Si verifica inoltre che lo stesso venga letto dal database correttamente,ottenendo un oggetto identico a quello di partenza   } & Superato & \parbox[t]{\dimTipo} { SubjectDAO() \\ existSubjectlWithName() \\ getSubjectByName() \\ getAllSubject() \\ getAllSubjectName() \\ createSubject() \\ subjectsOfGroup() \\ getGroupOfSubject() \\ getSubjectsByType() \\ } \\
\hline
TU7 & \parbox[t]{\larghezza}{ Si verifica che dato un GroupOfSubject ben formato,esso venga correttamente salvato nel database.Si verifica inoltre che lo stesso venga letto dal database correttamente,ottenendo un oggetto identico a quello di partenza  } & Superato & \parbox[t]{\dimTipo} { GroupDAO() \\ existGroupWithName() \\ getAllGroup() \\ getGroupByName() \\ createGroup() \\ deleteGroup() \\ addSubjectToGroup() \\ removeSubjectFromGroup() \\ getSubjectOfGroup() \\ } \\
\hline
TU8 & \parbox[t]{\larghezza}{ Si verifica che dato un Dataset ben formato,esso venga correttamente salvato nel database.Si verifica inoltre che lo stesso venga letto dal database correttamente,ottenendo un oggetto identico a quello di partenza  } & Superato & \parbox[t]{\dimTipo} { DatasetDAO() \\ existSubjectWithName() \\ getAllDataset() \\ getAllDatasetName() \\ getGroupOfDataset() \\ getProtocolOfDataset() \\ createDataset() \\ deleteDataset() \\ addProtocol() \\ } \\
\hline
TU9 & \parbox[t]{\larghezza}{ Si verifica che dato un Protocol ben formato,esso venga correttamente salvato nel database.Si verifica inoltre che lo stesso venga letto dal database correttamente,ottenendo un oggetto identico a quello di partenza  } & Superato & \parbox[t]{\dimTipo} { ProtocolDAO() \\ existProtocolWithName() \\ getAllProtocol() \\ getProtocolByName() \\ createProtocol() \\ deleteProtocol() \\ getAllProtocolName() \\ getAllProtocolOfDataset() \\ } \\
\hline
TU10 & \parbox[t]{\larghezza}{ Si verifica che dato un Algorithm ben formato,esso venga correttamente salvato nel database.Si verifica inoltre che lo stesso venga letto dal database correttamente,ottenendo un oggetto identico a quello di partenza  } & Superato & \parbox[t]{\dimTipo} { AlgorithmDAO() \\ getAlgorithmById() \\ addAlgorithm() \\ deleteAlgorithm() \\ getAllAlgorithm() \\ getAlgorithmOfProtocol() \\ } \\
\hline
TU11 & \parbox[t]{\larghezza}{ Si verifica che dato un Analysis ben formato,esso venga correttamente salvato nel database.Si verifica inoltre che lo stesso venga letto dal database correttamente,ottenendo un oggetto identico a quello di partenza  } & Superato & \parbox[t]{\dimTipo} { AnalysisDAO() \\ createAnalysis() \\ deleteAnalysis() \\ getAllAnalysis() \\ getAnalysisByDate() \\ getAnalysisOfDataset() \\ } \\
\hline
TU12 & \parbox[t]{\larghezza}{ Si verifica che,data un InternalImage2D venga applicata correttamente,la feature extractor Standard Deviation.Per fare ciò viene confrontato il risultato,con l'immagine attesa,fornita direttamente dal proponente stesso. } & Superato & \parbox[t]{\dimTipo} { FirstOrderFeature \\ getWindowSize() \\ getParameters() \\ getType() \\ setParameters() \\ StandardDeviationFeature() \\ singleChannelExecution2D() \\ } \\
\hline
TU13 & \parbox[t]{\larghezza}{ Si verifica che,data un InternalImage3D venga applicata correttamente,la feature extractor Standard Deviation.Per fare ciò viene confrontato il risultato,con l'immagine attesa,fornita direttamente dal proponente stesso. } & Superato & \parbox[t]{\dimTipo} { FirstOrderFeature \\ getWindowSize() \\ getParameters() \\ getType() \\ setParameters() \\ StandardDeviationFeature() \\ singleChannelExecution3D() \\ } \\
\hline
TU14 & \parbox[t]{\larghezza}{ Si verifica che,data un InternalImage2D venga applicata correttamente,la feature extractor Skewness.Per fare ciò viene confrontato il risultato,con l'immagine attesa,fornita direttamente dal proponente stesso. } & Superato & \parbox[t]{\dimTipo} { FirstOrderFeature \\ getWindowSize() \\ getParameters() \\ getType() \\ setParameters() \\ SkewnessFeature() \\ singleChannelExecution2D() \\ } \\
\hline
TU15 & \parbox[t]{\larghezza}{ Si verifica che,data un InternalImage3D venga applicata correttamente,la feature extractor Skewness.Per fare ciò viene confrontato il risultato,con l'immagine attesa,fornita direttamente dal proponente stesso. } & Superato & \parbox[t]{\dimTipo} { FirstOrderFeature \\ getWindowSize() \\ getParameters() \\ getType() \\ setParameters() \\ SkewnessFeature() \\ singleChannelExecution3D() \\ } \\
\hline
TU16 & \parbox[t]{\larghezza}{ Si verifica che,data un InternalImage2D venga applicata correttamente,la feature extractor Kurtosis.Per fare ciò viene confrontato il risultato,con l'immagine attesa,fornita direttamente dal proponente stesso. } & Superato & \parbox[t]{\dimTipo} { FirstOrderFeature \\ getWindowSize() \\ getParameters() \\ getType() \\ setParameters() \\ KurtosisFeature() \\ singleChannelExecution2D() \\ } \\
\hline
TU17 & \parbox[t]{\larghezza}{ Si verifica che,data un InternalImage3D venga applicata correttamente,la feature extractor Kurtosis.Per fare ciò viene confrontato il risultato,con l'immagine attesa,fornita direttamente dal proponente stesso. } & Superato & \parbox[t]{\dimTipo} { FirstOrderFeature \\ getWindowSize() \\ getParameters() \\ getType() \\ setParameters() \\ KurtosisFeature() \\ singleChannelExecution3D() \\ } \\
\hline
TU18 & \parbox[t]{\larghezza}{ Si verifica che,data un InternalImage2D venga applicata correttamente,la feature extractor Mean.Per fare ciò viene confrontato il risultato,con l'immagine attesa,fornita direttamente dal proponente stesso. } & Superato & \parbox[t]{\dimTipo} { FirstOrderFeature \\ getWindowSize() \\ getParameters() \\ getType() \\ setParameters() \\ MeanFeature() \\ singleChannelExecution2D() \\ } \\
\hline
TU19 & \parbox[t]{\larghezza}{ Si verifica che,data un InternalImage3D venga applicata correttamente,la feature extractor Mean.Per fare ciò viene confrontato il risultato,con l'immagine attesa,fornita direttamente dal proponente stesso. } & Superato & \parbox[t]{\dimTipo} { FirstOrderFeature \\ getWindowSize() \\ getParameters() \\ getType() \\ setParameters() \\ MeanFeature() \\ singleChannelExecution3D() \\ } \\
\hline
TU20 & \parbox[t]{\larghezza}{ Si verifica che,data un InternalImage2D venga applicata correttamente,la feature extractor Correlation.Per fare ciò viene confrontato il risultato,con l'immagine attesa,fornita direttamente dal proponente stesso. } & Superato & \parbox[t]{\dimTipo} { SecondOrderFeature() \\ getWindowSize() \\ getParameters() \\ getType() \\ setParameters()  \\ CorrelationFeature() \\ singleChannelExecution2D() \\ } \\
\hline
TU21 & \parbox[t]{\larghezza}{ Si verifica che,data un InternalImage3D venga applicata correttamente,la feature extractor Correlation.Per fare ciò viene confrontato il risultato,con l'immagine attesa,fornita direttamente dal proponente stesso. } & Superato & \parbox[t]{\dimTipo} { SecondOrderFeature() \\ getWindowSize() \\ getParameters() \\ getType() \\ setParameters()  \\ CorrelationFeature() \\ singleChannelExecution3D() \\ } \\
\hline
TU22 & \parbox[t]{\larghezza}{ Si verifica che,data un InternalImage2D venga applicata correttamente,la feature extractor Energy.Per fare ciò viene confrontato il risultato,con l'immagine attesa,fornita direttamente dal proponente stesso. } & Superato & \parbox[t]{\dimTipo} { SecondOrderFeature() \\ getWindowSize() \\ getParameters() \\ getType() \\ setParameters()  \\ EnergyFeature() \\ singleChannelExecution2D() \\ } \\
\hline
TU23 & \parbox[t]{\larghezza}{ Si verifica che,data un InternalImage3D venga applicata correttamente,la feature extractor Energy.Per fare ciò viene confrontato il risultato,con l'immagine attesa,fornita direttamente dal proponente stesso. } & Superato & \parbox[t]{\dimTipo} { SecondOrderFeature() \\ getWindowSize() \\ getParameters() \\ getType() \\ setParameters()  \\ EnergyFeature() \\ singleChannelExecution3D() \\ } \\
\hline
TU24 & \parbox[t]{\larghezza}{ Si verifica che,data un InternalImage2D venga applicata correttamente,la feature extractor Contrast.Per fare ciò viene confrontato il risultato,con l'immagine attesa,fornita direttamente dal proponente stesso. } & Superato & \parbox[t]{\dimTipo} { SecondOrderFeature() \\ getWindowSize() \\ getParameters() \\ getType() \\ setParameters()  \\ ContrastFeature() \\ singleChannelExecution2D() \\ } \\
\hline
TU25 & \parbox[t]{\larghezza}{ Si verifica che,data un InternalImage3D venga applicata correttamente,la feature extractor Contrast.Per fare ciò viene confrontato il risultato,con l'immagine attesa,fornita direttamente dal proponente stesso. } & Superato & \parbox[t]{\dimTipo} { SecondOrderFeature() \\ getWindowSize() \\ getParameters() \\ getType() \\ setParameters() \\ ContrastFeature() \\ singleChannelExecution3D() \\ } \\
\hline
TU26 & \parbox[t]{\larghezza}{ Si verifica che,data un InternalImage2D venga applicata correttamente,la feature extractor Homogeneity.Per fare ciò viene confrontato il risultato,con l'immagine attesa,fornita direttamente dal proponente stesso. } & Superato & \parbox[t]{\dimTipo} { SecondOrderFeature() \\ getWindowSize() \\ getParameters() \\ getType() \\ setParameters() \\ HomogeneityFeature() \\ singleChannelExecution2D() \\ } \\
\hline
TU27 & \parbox[t]{\larghezza}{ Si verifica che,data un InternalImage3D venga applicata correttamente,la feature extractor Homogeneity.Per fare ciò viene confrontato il risultato,con l'immagine attesa,fornita direttamente dal proponente stesso. } & Superato & \parbox[t]{\dimTipo} { SecondOrderFeature() \\ getWindowSize() \\ getParameters() \\ getType() \\ setParameters() \\ HomogeneityFeature () \\ singleChannelExecution3D() \\ } \\
\hline
TU28 & \parbox[t]{\larghezza}{ Si verifica che,data un InternalImage2D venga applicata correttamente,la feature extractor Entropy.Per fare ciò viene confrontato il risultato,con l'immagine attesa,fornita direttamente dal proponente stesso. } & Superato & \parbox[t]{\dimTipo} { SecondOrderFeature() \\ getWindowSize() \\ getParameters() \\ getType() \\ setParameters() \\ EntropyFeature() \\ singleChannelExecution2D() \\ } \\
\hline
TU29 & \parbox[t]{\larghezza}{ Si verifica che,data un InternalImage3D venga applicata correttamente,la feature extractor Entropy.Per fare ciò viene confrontato il risultato,con l'immagine attesa,fornita direttamente dal proponente stesso. } & Superato & \parbox[t]{\dimTipo} { SecondOrderFeature() \\ getWindowSize() \\ getParameters() \\ getType() \\ setParameters() \\ EntropyFeature() \\ singleChannelExecution3D() \\ } \\
\hline
TU30 & \parbox[t]{\larghezza}{ Si verifica che,data un InternalImage2D venga applicato correttamente,l' algoritmo di clustering Fuzzy C Mean.Per fare ciò viene confrontato il risultato ottenuto,con l'immagine attesa,fornita direttamente dal proponente stesso. } & Superato & \parbox[t]{\dimTipo} { FuzzyCMeansAlgorithm() \\ singleChannelExecution2D() \\ getNumberOfCluster() \\ getFuzzyIndex() \\ getMaxIteration() \\ getThreshold() \\ } \\
\hline
TU31 & \parbox[t]{\larghezza}{ Si verifica che,data un InternalImage3D venga applicato correttamente,l' algoritmo di clustering Fuzzy C Mean.Per fare ciò viene confrontato il risultato ottenuto,con l'immagine attesa,fornita direttamente dal proponente stesso. } & Superato & \parbox[t]{\dimTipo} { FuzzyCMeansAlgorithm() \\ singleChannelExecution3D() \\ getNumberOfCluster() \\ getFuzzyIndex() \\ getMaxIteration() \\ getThreshold() \\ } \\
\hline
TU32 & \parbox[t]{\larghezza}{ Si verifica che,data un InternalImage2D venga applicato correttamente,l' algoritmo di clustering Hierarchical.Per fare ciò viene confrontato il risultato ottenuto,con l'immagine attesa,fornita direttamente dal proponente stesso. } & Superato & \parbox[t]{\dimTipo} { HiererchicalAlgorithm() \\ singleChannelExecution2D() \\ getDistance() \\ getLinkageCriteria() \\ } \\
\hline
TU33 & \parbox[t]{\larghezza}{ Si verifica che,data un InternalImage3D venga applicato correttamente,l' algoritmo di clustering Hierarchical.Per fare ciò viene confrontato il risultato ottenuto,con l'immagine attesa,fornita direttamente dal proponente stesso. } & Superato & \parbox[t]{\dimTipo} { HiererchicalAlgorithm() \\ singleChannelExecution3D() \\ getDistance() \\ getLinkageCriteria() \\ } \\
\hline
TU34 & \parbox[t]{\larghezza}{ Si verifica che,data un InternalImage2D venga applicato correttamente,l' algoritmo di clustering KMeans.Per fare ciò viene confrontato il risultato ottenuto,con l'immagine attesa,fornita direttamente dal proponente stesso. } & Superato & \parbox[t]{\dimTipo} { KMeansAlgorithm() \\ singleChannelExecution2D() \\ getNumberOfCluster() \\ getNumberOfReplicates() \\ getMaxIteration() \\ getDistance() \\ } \\
\hline
TU35 & \parbox[t]{\larghezza}{ Si verifica che,data un InternalImage3D venga applicato correttamente,l' algoritmo di clustering KMeans.Per fare ciò viene confrontato il risultato ottenuto,con l'immagine attesa,fornita direttamente dal proponente stesso. } & Superato & \parbox[t]{\dimTipo} { KMeansAlgorithm() \\ singleChannelExecution3D() \\ getNumberOfCluster() \\ getNumberOfReplicates() \\ getMaxIteration() \\ getDistance() \\ } \\
\hline
TU36 & \parbox[t]{\larghezza}{ Si verifica che,dato un percorso del filesystem,e delle stringhe di prova,venga creato un file di log.
In tale file devono essere stampate le stringhe attese rispetto a quelle date in input  } & Superato & \parbox[t]{\dimTipo} { Log() \\ countLines() \\ logWriter() \\ setEnabled() \\ writeLog() \\ } \\
\hline
TU37 & \parbox[t]{\larghezza}{ Si verifica che data una Feature ben formata con dei parametri corretti,essa venga correttamente salvata nel database.Si verifica inoltre che la stessa venga letta dal database correttamente,ottenendo un oggetto identico a quello di partenza } & Superato & \parbox[t]{\dimTipo} { FeatureDAO() \\ createFeature() \\ removeFeature() \\ getFeatureOfProtocol() \\ getAllFeatureOfProtocol() \\ } \\
\hline
TU38 & \parbox[t]{\larghezza}{ Get-Set } & Superato & \parbox[t]{\dimTipo} { setToolBar() \\ getTypeRomeo() \\ selectAll() \\ setSubjectName() \\ setImagePath() \\ setMaskPath() \\ setTypeSelection() \\ getSubjectName() \\ getMaskPath() \\ getTypeSelected() \\ getTableView \\ getType() \\ getSelectedSubjects() \\ setNSubjects \\ getTypeSelected() \\ getAlgorithmSelected() \\ getProtocolName() \\ getFeaturesTable() \\ getGroupTable() \\ getProtocolsTable() \\ getGroupInfoLabel() \\ getProtocolInfoLabel() \\ setProtocolInfo() \\ setGroupInfo() \\ getTableView() \\ setSubjectImg() \\ setSubjectMask() \\ setSubjectInfo() \\ getGroup() \\ getSubjectsTable() \\ getAlgorithmLabel() \\ getFeatureLabel() \\ setFeature() \\ getFeature() \\ getControllerManger() \\ setEditFields() \\ getMainWindowController() \\ } \\
\hline
TU39 & \parbox[t]{\larghezza}{ Architettura di sistema } & Superato & \parbox[t]{\dimTipo} { loadCss() \\ setupMainWindow() \\ addCentralWidget() \\ loadCssStatic() \\ setupLayout() \\ loadCss() \\ setupObjectName() \\ setupToolTip() \\ addConnect() \\ setupView() \\ setupLeftFrame() \\ setupRightFrame() \\ loadCss() \\ setupToolTip() \\ addConnect() \\ setupView() \\ setupLayout() \\ createTop() \\ createButtom() \\ loadCss() \\ setupObjectName() \\ setupToolTip() \\ addConnect() \\ setupView() \\ setupTopLayout() \\ createTop() \\ createButtom() \\ setupLayout() \\ loadCss() \\ setupObjectName() \\ setupToolTip() \\ addConnect() \\ setupView() \\ createTop() \\ createButtom() \\ loadCss() \\ setupObjectName() \\ setupToolTip() \\ addConnect() \\ setupView() \\ createCenter() \\ setupTopLayout() \\ setupLayout() \\ setupLayout() \\ createTop() \\ createCenter() \\ createButtom() \\ setupTopLayout() \\ setupCenterLayout() \\ loadCss() \\ setupObjectName() \\ setupToolTip() \\ addConnect() \\}\\

& & & \parbox[t]{\dimTipo} {  setupView() \\ setupLayout() \\ createTop() \\ createButtom() \\ setupTable() \\ setupInfoBlock() \\ loadCss() \\ setupToolTip() \\ addConnect() \\ setupView() \\ itemSelected() \\ setupLayout() \\ createTop() \\ createButtom() \\ setupTopLayout() \\ loadCss() \\ setupObjectName() \\ setupToolTip() \\ addConnect() \\ setupView() \\ itemSelected() \\ setupLayout() \\ createTop() \\ createButton() \\ setupTopLayout() \\ loadCss() \\ setupObjectName() \\ setupToolTip() \\ addConnect() \\ setupView() \\ createButtom() \\ addConnect() \\ setupLayout() \\ setupTopLayout() \\ loadCss() \\ setupObjectName() \\ setupToolTip() \\ setupView() \\ setupLayout() \\ createTop() \\createButtom() \\ setupTopLayout() \\ loadCss() \\ setupObjectName() \\ setupToolTip() \\ addConnect() \\ setupView() \\ setupLayout() \\ createTop() \\ createButtom() \\ setupTopLayout() \\ loadCss() \\ setupObjectName() \\ setupToolTip() \\ addConnect() \\  }\\

& & & \parbox[t]{\dimTipo} {setupView() \\ setupLayout() \\ loadCss() \\ setupObjectName() \\ setupToolTip() \\ addConnect() \\ setupView() \\ createTop() \\ createButtom() \\ loadCss() \\ setupObjectName() \\ setupToolTip() \\ addConnect() \\ createTop() \\ createButtom() \\ } \\
\hline

TU40 & \parbox[t]{\larghezza}{ Signal-Slot } & Superato & \parbox[t]{\dimTipo} { slotOk() \\ slotHelp() \\ slotBack() \\ slotChangeSubjectName() \\ slotChangeSubjectImage() \\ slotChangeSubjectMask() \\ slotSaveSubject() \\ slotAddImage() \\ slotAddMask() \\ slotChangeType() \\ slotSelectAll() \\ slotDeselectAll() \\ slotTypeSelection() \\ slotSaveGroup() \\ slotBack() \\ slotAddFeatureClicked() \\ slotFeatureSelected() \\ slotOk() \\ slotTypeChanged() \\ slotGroupSelected() \\ slotProtocolSelected() \\ sloSaveDataset() \\ slotNewSubject() \\ slotShowSubjects() \\ slotNewGroup() \\ slotShowGroups() \\ slotNewProtocol() \\ slotShowProtocols() \\ slotNewDataset() \\ slotShowDatasets() \\ slotAboutRomeo() \\ slotShowHelp() \\ slotAnalysisResults() \\ slotStartAnalysis() \\ slotWelcome() \\ slotItemSelected() \\ slotItemSelected() \\ slotDeleteGroup() \\ slotEditGroup() \\ slotProtocolSelected() \\ }\\

& & &\parbox[t]{\dimTipo} {slotDeleteProtocol() \\ slotViewResult() \\ slotExportResult() \\ slotSelectedTreeItem() \\ slotSubjetOrder() \\ slotProtocolOrder() \\ slotShowAll() \\ slotShowResults() \\ slotSelectResult() \\ slotNextPage() \\ slotPreviousPage() \\ slotStartAnalysis() \\ slotSelectAllFeatures() \\ slotDeselectAllFeatures() \\ slotDatasetSelected() \\ slotExitFromAnalysis() \\ slotContinueAnalysis() \\ slotSelectOutputDirectory() \\ } \\
\hline
\caption{Descrizione test di unità }
\end{longtable}
\end{center}
