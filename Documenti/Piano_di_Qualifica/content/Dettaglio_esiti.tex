\section{Dettaglio dell'esito delle revisioni}
A seguito di ogni revisione, il committente pubblicherà una valutazione sull'andamento del progetto e sui documenti consegnati. Il gruppo quindi dovrà procedere ad una correzione laddove questa sia necessaria, in modo tale da proseguire i lavori su una base solida e corretta.
Di seguito sono elencate le modifiche effettuate a seguito delle revisioni.

\subsection{Revisione dei Requisiti}
\begin{itemize}
 \item \textbf{Norme di Progetto:} 
 	\begin{itemize}
 		\item sono stati inseriti vari diagrammi di flusso per rendere la lettura del documento più agevole e facilitarne la comprensione;
 		\item la struttura del documento è stata cambiata, suddividendolo per procedure, attività, processi e strumenti;
 		\item sono state integrate le parti riguardanti le tecniche di analisi, erroneamente descritte nel \textit{Piano di Qualifica}.
 	\end{itemize}
 	
 \item \textbf{Piano di Progetto:} 
 	\begin{itemize}
 		\item la qualità delle immagini è stata migliorata;
 		\item corretto l'utilizzo della terminologia.
 	\end{itemize}
 \item \textbf{Piano di Qualifica:}
 	\begin{itemize}
 		\item come suggerito, le parti riguardanti le tecniche adottate, sono state spostate nelle \textit{Norme di Progetto};
 		\item è stata aggiunta la sezione \textit{Definizione Obiettivi}.
 	\end{itemize}
 \item \textbf{Analisi dei Requisiti:}
 	\begin{itemize}
		\item sono state apportate le modifiche suggerite, migliorando e descrivendo i casi d'uso;
		\item aggiunti i requisiti di qualità;
		\item corretti alcuni requisiti.
 	\end{itemize}
\end{itemize}

\subsection{Revisione di Progettazione}
\begin{itemize}
\item \textbf{Norme di Progetto:}
	\begin{itemize}
		\item sono stati inseriti i meccanismi di controllo e rendicontazione, erroneamente descritti nel \textit{Piano di Progetto};
		\item è stata sistemata la struttura del documento suddividendo i contenuti sulla base dei processi che si è deciso di adottare.		
	\end{itemize}
\item \textbf{Piano di Progetto:}
	\begin{itemize}
		\item modificata la struttura del documento integrando alcuni argomenti semanticamente correlati, che prima risultavano slegati tra di loro;
		\item attualizzata la sessione riguardante l'analisi dei rischi;
		\item esplicitato il preventivo a finire;
		\item corretto l'utilizzo della terminologia, particolare attenzione va data al nome delle fasi che prima sviavano confondendo le fasi, con le attività in esse svolte.
	\end{itemize}
\item \textbf{Piano di Qualifica:}
	\begin{itemize}
		\item aggiunte le metriche riguardanti i processi;
		\item adattato lo standard ISO/IEC 15504 ai processi che vengono adottati.
	\end{itemize}
\item \textbf{Analisi dei Requisiti:}
	\begin{itemize}
		\item incrementata sezione riguardante i diagrammi delle attività;
		\item sistemati i casi d'uso non corretti segnalati dal committente;
		\item sistemata tipologia del requisito segnalato dal committente.
	\end{itemize}
\item  \textbf{Specifica Tecnica:}
	\begin{itemize}
		\item migliorata la contestualizzazione dei design pattern\glossario{};
		\item descrizione più approfondita delle relazioni tra le componenti logiche;
		\item sistemati i diagrammi delle classi per farli aderire ai design pattern\glossario{} associati;
		\item sistemati i diagrammi delle classi evidenziando le classi astratte e le interfacce;
		\item sistemata terminologia riguardante il database.
	\end{itemize}
\end{itemize}
%%%%%%%%%%%%%%%%%%%%%%%%%%%%%%%%%%%%%%%%%%%%%%%%%%%%%%%%%%
\subsection{Revisione di Qualifica}
\begin{itemize}
\item \textbf{Norme di Progetto:}
	\begin{itemize}
		\item inserite le tecniche di verifica in un nuovo capitolo \textit{Processo di Verifica} erroneamente descritti nel \textit{\textbf{Piano di Qualifica}};
		\item inseriti i contenuti riguardanti la gestione amministrativa della revisione erroneamente descritta nel \textit{\textbf{Piano di Qualifica}};
		\item integrati i contenuti del precedente capitolo 2 all'interno del processi processi organizzativi;
		\item suddivisi i contenuti del precedente capitolo 6 in relazione alle attività che supportano.	
	\end{itemize}
\item \textbf{Piano di Progetto:}
	\begin{itemize}
		\item esplicitato il preventivo a finire.
	\end{itemize}
\item \textbf{Piano di Qualifica:}
	\begin{itemize}
		\item esplicitati gli obiettivi riguardanti le metriche descritte nel capitolo \textit{Misure e metriche di qualità}.
	\end{itemize}
\item \textbf{Analisi dei Requisiti:}
	\begin{itemize}
		\item sistemata la grandezza dell'immagine come segnalato dal committente.
	\end{itemize}
\item  \textbf{Specifica Tecnica:}
	\begin{itemize}
		\item migliorata la contestualizzazione del design pattern\glossario{} DAO;
		\item sistemate le problematiche legate alla progettazione di alcune componenti.
	\end{itemize}
\item  \textbf{Definizione di Prodotto:}
	\begin{itemize}
		\item migliorata la contestualizzazione dei design pattern\glossario{};
		\item sistemata la terminologia da Package\g{} a Diagramma dei Package\g{};		
		\item sistemati gli errori commessi nella rappresentazione dei diagrammi delle classi non conforme allo standard UML 2.0;
		\item sistemati i diagrammi di sequenza in modo da favorire maggiormente la lettura;
		\item dettagliata maggiormente la descrizione dei metodi segnalati dal committente in sede di revisione di qualifica;
		\item sistemate incongruenze tra la descrizione dei metodi e la loro rappresentazione nei diagrammi delle classi. 
	\end{itemize}
\item  \textbf{Manuale Utente:}
	\begin{itemize}
		\item è stato inserito la procedura di installazione del prodotto;
		\item è stata descritta con maggior dettaglio l'esecuzione di un'analisi come segnalato in sede di revisione di qualifica.
	\end{itemize}
\end{itemize}
%\subsubsection{Revisione di Accettazione}
