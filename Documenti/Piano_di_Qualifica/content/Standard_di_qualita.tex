\section{Standard di qualità}

\subsection{Standard ISO/IEC 15504}
\label{std15504}
Lo standard ISO\glossario/ IEC\glossario 15504, definisce un modello per la valutazione della \textit{maturità} dei processi, indicando una metrica per misurarla. La scala di misura fornita dallo standard, si basa sul principio che un'alta \textit{capacità di processo}\glossario{} è associata ad un miglioramento prestazionale dello stesso.
\\Per ogni processo, si definisce un livello di capacità, basato sulla seguente scala a sei livelli, ognuno con i propri attributi:
\begin{itemize}
	\item \textbf{Livello 0 - Incomplete process:} il processo non raggiunge i risultati aspettati; non ci sono prodotti identificabili come risultati del processo;
	\item \textbf{Livello 1 - Performed process:} lo scopo del processo viene parzialmente raggiunto. Il risultato però, potrebbe non essere stato rigorosamente pianificato e tracciato; si riescono tuttavia ad identificare i prodotti del processo
		\begin{itemize}
	 		\item \textit{Process Performance}: capacità di un processo di raggiungere gli obiettivi e di rendere identificabili i suoi risultati.
	 	\end{itemize}
	 \item \textbf{Livello 2 - Managed process:} il processo produce prodotti di qualità, rispettando i tempi stabiliti. I prodotti sono conformi agli standard ed ai requisiti. La principale differenza con il livello precedente, consiste nella pianificazione delle prestazioni del prodotto attraverso processi definiti
		\begin{itemize}
	 		\item \textit{Performance Management:} misura la capacità del processo di produrre prodotti entro il tempo prestabilito e le risorse disponibili;
	 		\item \textit{Work Product Management:} misura la capacità del processo di produrre prodotti documentati, controllati e verificati.
	 	\end{itemize}
	 \item \textbf{Livello 3 - Established process:} il processo viene eseguito rispettando i principi dell'ingegneria del software. La principale differenza con il livello precedente consiste nell'adozione di un \textit{processo standard} per pianificare e gestire i processi
		\begin{itemize}
	 		\item \textit{Process Definition:} è la misura in cui l'esecuzione del processo aderisce al processo standard prefissato;
	 		\item \textit{Process Deployment:} è la misura in cui l'esecuzione del processo usa efficacemente adeguate risorse umane e tecniche.
	 	\end{itemize}
	 \item \textbf{Livello 4 - Predictable process:} il processo viene eseguito costantemente entro definiti limiti di controllo e vengono misurate dettagliatamente le prestazioni dello stesso; la qualità dei prodotti del processo è quantitativamente nota
		\begin{itemize}
	 		\item \textit{Process Measurament:} le misure ricavate dal processo vengono utilizzate per garantire il raggiungimento degli obbiettivi stabiliti;
	 		\item \textit{Process Control:} il processo viene controllato per effettuare eventuali correzioni e miglioramenti.
	 	\end{itemize}
	\item \textbf{Livello 5 - Optimizing process:} le prestazioni del processo sono ottimali e raggiunge la ripetibilità nel conseguimento degli suoi obbiettivi
	 	\begin{itemize}
	 		\item \textit{Process Change:} l'introduzione di cambiamenti deve minimizzare il rischio di eventuali peggioramenti nel processo;
	 		\item \textit{Continuous Improvement:} si adotta un approccio proattivo nell'identificare eventuali modifiche per migliorare i processi esistenti.
	 	\end{itemize}
\end{itemize}
Ogni attributo è misurabile e lo standard fissa la seguente scala di valutazione:
\begin{itemize}
	\item \textbf{Not achieved} (0-15\%);
	\item \textbf{Partially achieved} (>15\%-50\%);
	\item \textbf{Largely achieved} (>50\%-85\%);
	\item \textbf{Fully achieved} (85\%-100\%).
\end{itemize}

\subsection{Standard ISO/IEC 9126}
\label{std9126}
Lo standard ISO\glossario{}/IEC\glossario{}9126 delinea una serie di normative atte a descrivere un modello di qualità del software. Il modello propone un approccio alla qualità focalizzato a favorire un miglioramento dell'organizzazione e dei processi e, come conseguenza concreta, della qualità del prodotto sviluppato.
\\Il modello di qualità stabilito dallo standard è classificato in base a sei caratteristiche generali:
\begin{itemize}
	\item[\textbf{1.}]\textbf{Funzionalità:} è la capacità del prodotto software di fornire soluzioni che soddisfino esigenze stabilite
	\begin{itemize}
		\item \textbf{Appropriatezza:} il prodotto software fornisce un appropriato insieme di funzioni per i compiti specificati e gli obiettivi prefissati;
		\item \textbf{Accuratezza:} il prodotto software fornisce i risultati richiesti;
		\item \textbf{Interoperabilità:} il prodotto software interagisce ed opera con uno o più sistemi specificati;
		\item \textbf{Conformità:} il prodotto software aderisce a standard, convenzioni e regolamentazioni rilevanti al settore operativo a cui vengono applicate;
		\item \textbf{Sicurezza:} il prodotto software protegge le informazioni e i dati contenuti, negando che persone o sistemi non autorizzati possano accedervi o modificarli e li mette a disposizione dei soli autorizzati.
	\end{itemize}
	\item[\textbf{2.}]\textbf{ Affidabilità:} è la capacità del prodotto software di mantenere uno specificato livello di prestazioni
	\begin{itemize}
		\item \textbf{Maturità:} non si verificano errori e malfunzionamenti durante l'uso del prodotto;
		\item \textbf{Tolleranza agli errori:} il prodotto riesce a mantenere livelli prefissati di prestazioni anche in presenza di eventuali malfunzionamenti;
		\item \textbf{Recuperabilità:} il prodotto riesce a recuperare delle informazioni rilevanti a seguito di un malfunzionamento;
		\item \textbf{Aderenza:} il prodotto aderisce a standard, regole e convenzioni in ambito di affidabilità.
	\end{itemize}
	\item[\textbf{3.}]\textbf{Efficienza:} è la capacità di fornire prestazioni relativamente alla quantità di risorse usate
	\begin{itemize}
		\item \textbf{Comportamento rispetto al tempo:} i tempi di risposta del prodotto sono ragionevoli rispetto alla richiesta effettuata;
		\item \textbf{Efficienza:} l'utilizzo delle risorse è adeguato alle attività da svolgere;
		\item \textbf{Conformità:} il prodotto aderisce a standard, regole e convenzioni in ambito di affidabilità.
	\end{itemize}
	\item[\textbf{4.}]\textbf{Usabilità:} è la capacità del prodotto di essere capito, appreso e usato dall'utente
	\begin{itemize}
		\item \textbf{Comprensibilità:} esprime la facilità di comprensione delle funzionalità del prodotto;
		\item \textbf{Apprendibiltà:} l'uso prodotto risulta essere di facile apprendimento da parte degli utenti;
		\item \textbf{Operabilità:} capacità di mettere gli utenti in condizione di fare uso delle funzionalità del prodotto;
		\item \textbf{Attrattività:} il prodotto riesce ad attrarre gli utenti;
		\item \textbf{Conformità:} il prodotto aderisce a standard, regole e convenzioni in ambito di usabilità.
	\end{itemize}
	\item[\textbf{5.}]\textbf{Manutenibilità:} è la capacità del prodotto di essere manutenibile nel corso del tempo, al fine di consentire la correzione degli errori o il rilascio di nuove funzionalità
	\begin{itemize}
		\item \textbf{Analizzabilità:} indice di facilità con la quale è possibile analizzare il codice per localizzare eventuali errori;
		\item \textbf{Modificabilità:} capacità del prodotto di permettere l'implementazione di future modifiche;
		\item \textbf{Stabilità:} capacità di evitare effetti inaspettati derivanti da modifiche errate;
		\item \textbf{Testabilità:} il prodotto è facilmente testabile per validare le modifiche apportate.
	\end{itemize}
	\item[\textbf{6.}]\textbf{Portabilità:} è la capacità del software di operare in diversi ambienti di lavoro
	\begin{itemize}
		\item \textbf{Adattabilità:} il software si adatta a differenti ambienti senza dover applicare modifiche diverse da quelle fornite;
		\item \textbf{Installabilità:} capacità del software di essere installato in uno specifico ambiente;
		\item \textbf{Conformità:} il prodotto aderisce a standard, regole e convenzioni in ambito di usabilità;
		\item \textbf{Sostituibilità:} il prodotto riesce a sostituire i prodotti analoghi già esistenti.
	\end{itemize}
\end{itemize}