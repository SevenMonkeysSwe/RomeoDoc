%%%%%%%%%%%%%%%%%%%%%%%%%%%%%%%%%%%%%%%%%%%%%%%%%%%%%%%%%%%%%%%%%%%%%%%%%%%%%%%%%%%%%%%%%%%%%%%%%%%%%%%%%%%%%%%%%%
% File: commands	.tex
% Created: 2013/11/25
% Author: Magnabosco Nicola
% Email: nick.magnabosco@gmail.com
%----------------------------------------------------------------------------------------------------------------
% Modification History:
% Version		Modifier Date		Change													Author
% 1.0.3			2013-12-09			Add commands for the documents files.
% 1.0.2			2013/11/29			Correct command proposerName for misspelling				Magnabosco Nicola
% 1.0.1			2013/11/27			Correct command "analist" for misspelling				Magnabosco Nicola
% 1.0.0			2013/11/25			Add all Custom Commands									Magnabosco Nicola
%%%%%%%%%%%%%%%%%%%%%%%%%%%%%%%%%%%%%%%%%%%%%%%%%%%%%%%%%%%%%%%%%%%%%%%%%%%%%%%%%%%%%%%%%%%%%%%%%%%%%%%%%%%%%%%%%%

%%*******REMEMBER TO MODIFY THIS PART FOR ANY NEW DOCUMENT******
%%INSERIRE IL NOME DEL DOCUMENTO
\newcommand{\documentName}{NOME DEL DOCUMENTO}

%%INSERT THE CURRENT VERSION
\newcommand{\currentVersion}{VERSION}

%%INSERT HERE DOCUMENT DATE
\newcommand{\redactionDate}{\today}

%%INSERT HERE THE NAME OF THE EDITORS
\newcommand{\editors}{
	\begin{itemize}
		\item Tizio Caio
		\item Mario Rossi
	\end{itemize}}

%%INSERT THE NAME OF THE DOCUMENT VERIFIERS
\newcommand{\verifiers}{\emph{Verificatore}}

%%INSERT WHO APPROVE THE DOCUMENT
\newcommand{\approved}{\emph{..}}

\newcommand{\usage}{\emph{Interno/Esterno}}

%%DOCUMENT DISTRIBUTION LIST
\newcommand{\distributionList}{\begin{itemize}
\item \authorName
\item \commitName
\item \commitNameC
\item \proposerName
\end{itemize}}

%%INSERIRE IL SOMMARIO
\newcommand{\sommario}{Descrizione dello scopo del documento}

%%DEFINIZIONE GLOSSARIO UGUALE PER TUTTI NON MODIFICARE
\newcommand{\glossario}{\small\ped{\textbf{G}}}
\newcommand{\g}{\small{\ped{\textbf{G}}}}

% BASE COMMANDS - || DO NOT MODIFY ||
\newcommand{\project}{Romeo}
\newcommand{\email}{\textit{\href{mailto:seven.monkeys.swe@gmail.com}{seven.monkeys.swe@gmail.com}}}
\newcommand{\authorName}{\emph{Seven Monkeys}}
\newcommand{\projectManager}{\emph{Responsabile di Progetto}}
\newcommand{\administrator}{\emph{Amministratore di Progetto}}
\newcommand{\analyst}{\emph{Analista}}
\newcommand{\programmer}{\emph{Programmatore}}
\newcommand{\designer}{\emph{Progettista}}
\newcommand{\verifier}{\emph{Verificatore}}
\newcommand{\proposerName}{\emph{Dott.ssa Gaia Rizzo} - DEI}
\newcommand{\commitName}{\emph{Prof. Tullio Vardanega}}
\newcommand{\commitNameC}{\emph{Prof. Riccardo Cardin}}
\newcommand{\logo}{\includegraphics[width=4cm,height=4cm]{../img/big_logo}}
\newcommand{\data}{\today}
\newcommand{\glossDoc}{\emph{Glossario v3.0.0}}
\newcommand{\studio}{\emph{Studio di Fattibilità \currentVersion}}
\newcommand{\analisi}{\emph{Analisi dei Requisiti \currentVersion}}
\newcommand{\PdP}{\emph{Piano di Progetto \currentVersion}}
\newcommand{\PdQ}{\emph{Piano di Qualifica \currentVersion}}
\newcommand{\NdP}{\emph{Norme di Progetto \currentVersion}}
\newcommand{\scopoProd}
{Il prodotto che si intende realizzare, denominato \project{}, si propone di fornire un sistema software per applicare la cluster analysis\glossario{} ad immagini biomediche. Lo scopo principale è quello di offrire alla comunità scientifica internazionale uno strumento semplice, ma allo stesso tempo completo e flessibile per applicare gli algoritmi della cluster analysis\glossario{}.}
\newcommand{\glossIntro}
{Al fine di evitare ogni ambiguità e per permettere al lettore una migliore comprensione dei termini e acronimi utilizzati nei vari documenti formali, essi sono riportati nel \glossDoc{} che
 contiene una descrizione approfondita di tali termini e acronimi.
\\Ogni volta
 che compare un termine presente nel \emph{Glossario}, esso è marcato
 con una \lq\lq\textbf{G}\rq\rq{} in pedice.}
\newcommand{\subject}{Subject\g{}}
\newcommand{\protocol}{Protocol\g{}}
\newcommand{\dataset}{Dataset\g{}}
\newcommand{\scopoDocumentoSpecifica}{Il presente documento ha lo scopo di definire la progettazione ad alto livello dell'applicativo \project{}.
Verrà presentata l'architettura generale secondo la quale saranno organizzate le componenti software e saranno descritti i design pattern\glossario{} utilizzati.}
