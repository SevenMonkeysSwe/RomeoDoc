%%%%%%%%%%%%%%%%%%%%%%%%%%%%%%%%%%%%%%%%%%%%%%%
%Analisi Dei Rischi
%Version: 0.0.1
%Author: Nicolò Bissacco(nickbissa@gmail.com)
%Creation Date: 2013-12-11
%Change Table
%Relase		Author			Date		Object of the changed
%%%%%%%%%%%%%%%%%%%%%%%%%%%%%%%%%%%%%%%%%%%%%%%%
\section{Pianificazione}
La seguente sezione ha lo scopo di illustrare inizialmente i rischi che si potrebbero incontrare durante lo svolgimento del commitment preso e successivamente l'individuazione delle attività da svolgere e i ruoli coinvolti per la realizzazione.
\subsection{Analisi dei rischi}
\label{AnalisiDeiRischi}
	La gestione dei rischi si divide in quattro fasi:
	\begin{itemize}
		\item \textbf{Identificazione:} individuare i rischi connessi alla realizzazione del progetto, enuclearne le cause e prevedere le conseguenze che possono influenzare il prodotto, il progetto e il business;
		\item \textbf{Analisi:} prevedere la probabilità di occorrenza del rischio e quali potrebbero essere le possibili conseguenze;
		\item \textbf{Pianificazione:} formulare una strategia per evitare e affrontare eventuali rischi;
		\item \textbf{Monitoraggio:} elaborare metodi di verifica e controllo per il riconoscimento tempestivo dei rischi, il loro trattamento e la loro mitigazione.
	\end{itemize}
	I rischi identificati vengono etichettati con un nome, un livello di rischio, la probabilità che il rischio si verifichi, una descrizione e delle contromisure. Inoltre ad ogni rilascio dei documenti, verranno attualizzati, informando se si sono verificati, cos'hanno comportato e come si è agito per mitigarli.	
	\subsubsection{Rischio legato alle tecnologie}
	\label{AnalisiDeiRischi_Tecnologie}
		\textbf{Livello di rischio:} Basso
		\\ \\
		\textbf{Probabilità:} Bassa
		\\ \\
		\textbf{Descrizione:} Le tecnologie utilizzate sono conosciute in modo generale da tutti i componenti del gruppo, in quanto il linguaggio di programmazione utilizzato per lo sviluppo del capitolato è il C++\glossario. Nonostante ciò, alcuni aspetti avanzati del linguaggio saranno argomenti di studio.\\
		Un punto di criticità nelle tecnologie utilizzate è rappresentato dall'implementazione degli algoritmi dei feature extractors\glossario{} e della cluster analysis\glossario{}, che saranno applicati alle immagini in input per produrre l'output.
		\\ \\
		\textbf{Contromisure:} Per sopperire alle eventuali difficoltà correlate all'utilizzo del linguaggio e dei nuovi algoritmi utilizzati, ogni membro del gruppo ha il compito di approfondire la conoscenza attraverso lo studio individuale e una ricerca approfondita della documentazione, non sempre dettagliata, soprattutto sugli algoritmi da implementare.
		\\ \\
		\textbf{Attualizzazione:} Il rischio si è verificato a causa di un'errata fornitura delle informazioni da parte dei proponenti. Infatti ci è stato assicurato che alcune tecnologie presenti e alcuni linguaggi di programmazione consigliati, fossero stati adeguati e facilmente integrabili con il nostro prodotto. Questo ha portato ad un ritardo nella consegna della Revisione di Qualifica, in quanto il team ha dovuto fermare i lavori programmati, per poter studiare approfonditamente svariate soluzioni per sopperire a mancanze che le tecnologie offrivano o non adeguata integrazione con quanto si stava sviluppando. Il rischio sembra essere stato mitigato in maniera opportuna, e non dovrebbe creare altre lacune, quindi si presume che da qua, alla fine dei lavori, non si verifichi più.\\ Il rischio è stato inizialmente sottovalutato dando una stima di livello di rischio bassa, questo è dovuto a inesperienza da parte del gruppo in un progetto di questa portata; diversamente, il livello di rischio è stato abbastanza consistente in quanto ha portato ad un ritardo nelle consegne e ad una ripianificazione delle attività mancanti.
		
	\subsubsection{Rischio legato alle Risorse}
	\label{AnalisiDeiRischi_Personale}
		\textbf{Livello di rischio:} Medio
		\\ \\
		\textbf{Probabilità:} Alta
		\\ \\
		\textbf{Descrizione:} Ciascuno dei componenti del gruppo ha impegni e necessità personali che possono comportare un rallentamento nel regolare svolgimento dei compiti assegnati.\\
		Bisogna inoltre tenere conto di eventuali assenze o indisposizioni dovute a malattie.
		\\ \\
		\textbf{Contromisure:} Per ovviare le difficoltà sopracitate, è quindi indispensabile organizzare e coordinare al meglio tutte le attività tramite un'accurata pianificazione delle tempistiche e una ripartizione ottimale dei compiti.\\
		Bisogna inoltre prevedere l'interscambiabilità dei ruoli in maniera che di fronte ad eventuali assenze gli altri membri possano sostituirsi e proseguire la realizzazione del progetto.
		\\ \\
		\textbf{Attualizzazione:} Il rischio si è verificato durante il corso del progetto, dovuto principalmente a influenze di stagione. Questo non ha influito negativamente sul proseguimento del lavoro in quanto, la modifica delle scadenze ha dato la possibilità di recuperare il tempo perso dovuto a mancanze di più risorse nello stesso periodo di tempo. \\
		Interessi legati ad esami di altri corsi di studio ed esami mancanti, ha portato l'assenza di alcune risorse, che però, una volta terminato il periodo di interesse, si sono impegnate per portare a termine nei tempi stabiliti il lavoro a loro assegnato.
		
	\subsubsection{Rischio legato alla collaborazione}
	\label{AnalisiDeiRischi_Organizzazione}
		\textbf{Livello di rischio:} Medio
		\\ \\
		\textbf{Probabilità:} Alta
		\\ \\
		\textbf{Descrizione:} Il lavoro di un team numeroso in un progetto impegnativo rappresenta una nuova esperienza per tutti i componenti del gruppo. Gli attriti che normalmente potrebbero insorgere per le diversità personali, possono aumentare di fronte all'inesperienza generale e alle pressioni determinate dalle scadenze imposte.
		\\ \\
		\textbf{Contromisure:} Il \projectManager{} ha il compito di sorvegliare sull'operato degli altri membri del gruppo e redimere eventuali controversie che inevitabilmente nascono dal confronto diretto di persone diverse. È necessario, inoltre, mantenere acceso il confronto tra le parti per dipanare i dubbi, unificare le opinioni e allocare le risorse in modo diverso qualora il problema non si possa risolvere con un confronto civile.
		\\ \\
		\textbf{Attualizzazione:} Il rischio si è verificato in alcune occasioni, ma in modo contenuto. Non è stato dunque necessario un intervento pesante da parte del \projectManager{}; gli episodi si sono risolti con un confronto costruttivo tra le persone coinvolte.  
		 
	\subsubsection{Rischio legato alla modifica dei requisiti in corso d'opera}
	\label{AnalisiDeiRischi_Requisiti}
		\textbf{Livello di rischio:} Medio
		\\ \\
		\textbf{Probabilità:} Media
		\\ \\
		\textbf{Descrizione:} È possibile che alcuni requisiti presenti nel capitolato vengano interpretati in maniera diversa dalle richieste del proponente o che vengano modificati, aggiunti o rimossi in corso d'opera dal proponente.
		\\ \\		
		\textbf{Contromisure:} È opportuno che il gruppo effettui degli incontri con il proponente per fissare dei requisiti certi e dissolvere eventuali dubbi derivanti da requisiti complessi e dalle ambiguità presenti nel capitolato.
		\\ \\
		\textbf{Attualizzazione:} Il rischio si è verificato durante il primo periodo di sviluppo del capitolato dovuto al fatto che i proponenti non avevano ben chiari alcuni punti, e requisiti effettivamente necessari e quelli invece desiderabili, ma non fondamentali. Arrivati a questo punto di sviluppo la probabilità che si verifichi nuovamente il rischio è pressochè nulla, in quanto i requisiti individuati nel documento \analisi{} per il committente hanno raggiunto un grado di dettaglio buono inoltre gli stessi requisiti sono stati discussi precedentemente con i proponenti.
		
	\subsubsection{Rischio legato all'aumento dei costi}
	\label{AnalisiDeiRischi_Costi}
		\textbf{Livello di rischio:} Alto
		\\ \\
		\textbf{Probabilità:} Alta
		\\ \\
		\textbf{Descrizione:} È possibile che, per la complessità del progetto e le criticità di organizzazione sopracitate, i tempi preventivati in partenza possano subire un aumento che incide profondamente nel costo preventivato.
		\\ \\
		\textbf{Contromisure:} Per non eccedere la spesa preventivata e non ritardare i tempi di consegna previsti per l'ultimazione del progetto, nell'attività di pianificazione, si sono stabiliti, dei margini di tempo leggermente maggiorati rispetto alla realtà, soprattutto nella finalizzazione delle attività più critiche.\\
		Si sono inoltre pianificate delle scadenze a breve termine, con degli obiettivi minimi da raggiungere in tempi ristretti così da potere verificare, in itinere, lo stato di avanzamento dei lavori e poter aggiustare eventuali ritardi o deficit.
		\\ \\
		\textbf{Attualizzazione:} Il rischio non si è ancora verificato anche se ci sono stati dei ritardi nelle consegne. I ritardi però non hanno portato ad un aumento del costo preventivato in quanto si è riuscito a risparmiare del capitale dalle fasi precedenti e il ritardo era stato causato da una mancanza di conoscenza del dominio da parte dei componenti del gruppo, che, come detto precedentemente, hanno dovuto fermare i lavori, per poter dare ulteriore spazio alle ore per lo studio inerente al progetto e le opportunità offerte dall'ambiebte.
		
	\subsubsection{Rischio legato alla modifica delle scadenze}
	\label{modificaScadenze}
	\textbf{Livello di rischio:} Alto
	\\ \\
	\textbf{Probabilità:} Media
	\\ \\
	\textbf{Descrizione:} È possibile che durante lo svolgimento dei lavori, le Milestone\g{} esterne possano essere anticipate o posticipate rispetto alle date precedentemente comunicate. Questo potrebbe portare dei benefici (qualora ci siano dei ritardi e la Milestone\g{} posticipata) oppure  dei problemi (se la Milestone\g{} venisse anticipata) in quanto porterebbe a una ripianificazione delle attività in base ai tempi a disposizione, ma potrebbero non trovare lo spazio necessario per essere terminate entro la Milestone\g{} portando ad uno slittamento generale delle attività e dei lavori con conseguente aumento dei costi.
	\\ \\
	\textbf{Contromisure:} Per ridurre al massimo l'impatto negativo del rischio qualora si verificasse, occorre un'attenta ripianificazione delle attività, cercando di scomporre, qualora fosse possibile, le attività in parti più piccole, in modo tale da poterle assegnare ad altre risorse, in quel momento meno occupate riducendo al minimo i tempi \lq\lq{}morti\rq \rq delle risorse e riuscire a portare a termine i lavori nelle Milestone\g{} esterne prestabilite.
	\\ \\
	\textbf{Attualizzazione:} Il rischio si è verificato, e fortunatamente ha avuto un impatto positivo, in quanto, l'assenza di alcune risorse nello stesso periodo di tempo, ha portato ad un rallentamento dei lavori. Lo spostamento in avanti della Milestone\g{} ha dato la possibilità al gruppo di recuperare il ritardo accumulato, e di riuscire a portare a termine il lavoro, impiegando un po' del tempo rimanente a un controllo più accurato della documentazione prodotta.
	

