\newcommand{\tableHeader}{
		\begin{center}
		\begin{tabularx}{0.8\textwidth}{|X|X|X|}
			\hline
			\textbf{Nome} &
			\textbf{Data} &
			\textbf{Firma}\\ \hline
		}

\section{Organigramma}
\label{Organigramma}
\subsection{Redazione}
\begin{table} [!h]
	\tableHeader 
	Feltre Beatrice & 2013-12-09 & \includegraphics[height=0.5cm,width=3cm]{./content/Immagini/firme/bf.png}\\ \hline
	\end{tabularx}\end{center}
	\caption{Redazione}
\end{table}	

\subsection{Approvazione}
\begin{table} [!h]
	\tableHeader 
	Adami Alberto & 2013-12-09 & \includegraphics[height=0.5cm,width=3cm]{./content/Immagini/firme/aa.png}\\ \hline
	\end{tabularx}\end{center}
	\caption{Approvazione}
\end{table}	
\subsection{Accettazione componenti}
\begin{table}[!h]
	\tableHeader
	Adami Alberto & 2013-12-03 & 
		\includegraphics[height=0.5cm,width=3cm]{./content/Immagini/firme/aa.png}\\
	Bissacco Nicolò & 2013-12-03 & 
		\includegraphics[height=0.5cm,width=3cm]{./content/Immagini/firme/nb.png}\\
	Feltre Beatrice & 2013-12-03 & 
		\includegraphics[height=0.5cm,width=3cm]{./content/Immagini/firme/bf.png}\\
	Luisetto Luca & 2013-12-03 & 
		\includegraphics[height=0.5cm,width=3cm]{./content/Immagini/firme/ll.png}\\
	Magnabosco Nicola & 2013-12-03 & 
		\includegraphics[height=0.5cm,width=3cm]{./content/Immagini/firme/nm.png}\\
	Martignago Jimmy & 2013-12-03 & 
		\includegraphics[height=0.5cm,width=3cm]{./content/Immagini/firme/jm.png}\\
	Scapin Davide & 2013-12-03 & 
		\includegraphics[height=0.5cm,width=3cm]{./content/Immagini/firme/ds.png}\\ \hline
	\end{tabularx} \end{center}
	\caption{Accettazione Componenti}
\end{table}
\pagebreak
\subsection{Componenti}
\begin{table}[!h]
	\centering
	\begin{tabular}{|l|c|l|}
	\hline
	Cognome Nome & Matricola & Email\\ \hline
	Adami Alberto &1031350  & alberto.adami.7@gmail.com \\
	Bissacco Nicolò &572194&nickbissa@gmail.com  \\
	Feltre Beatrice &1001644 &beatrice.feltre@gmail.com \\
	Luisetto Luca &1006878&lucaluisetto91@gmail.com \\
	Magnabosco Nicola &1008149  &nick.magnabosco@gmail.com \\
	Martignago Jimmy &1002303&jimmy.martignago@gmail.com \\
	Scapin Davide &615296&davideskap@gmail.com\\ \hline
	\end{tabular}
	\caption{Componenti}
\end{table}
\subsection{Definizione dei ruoli}
\label{defRuoli}
Durante lo sviluppo di \project{} vi saranno diversi ruoli che i componenti del gruppo dovranno ricoprire. Ogni membro del gruppo, dovrà ricoprire almeno una volta ogni ruolo. È fatto divieto che una risorsa sia verificatrice di se stessa.
I ruoli individuati sono:
\begin{itemize}
\item \textbf{\projectManager:} rappresenta il gruppo, in quanto presenta al committente i risultati del progetto inoltre, accentra su di sé le responsabilità delle scelte e approvazioni, dunque:
	\begin{itemize}
	\item Pianifica, controlla e coordina le attività;
	\item Gestisce e controlla le risorse;
	\item Analizza i rischi che si possono avere e identifica contromisure per mitigarli;
	\item Approva i documenti, ad esclusione del \emph{Piano di Progetto};	
	\item Redige il \emph{Piano di Progetto}.	
	\end{itemize}
\item \textbf{\administrator:} è responsabile del funzionamento dell'ambiente di lavoro messo a disposizione, della sua efficienza e ne detiene il controllo.
I compiti rilevanti che gli competono sono:
	\begin{itemize}
	\item Ricerca di strumenti che possono automatizzare il più possibile le operazioni in modo tale da toglierle alla componente umana che potrebbe inserire errori;
	\item Risoluzione di problemi legati alla gestione e controllo di processi adottando strumenti adatti;
	\item Gestione e controllo del versionamento della documentazione riguardante il progetto;
	\item Fornire procedure strumenti e tutto il necessario per monitorare e segnalare il controllo qualità;
	\item Redazione delle \emph{Norme di Progetto} nel quale viene spiegato l'utilizzo degli strumenti adottati;
	\item Aiuto nella redazione del \emph{Piano di Qualifica}.
	\end{itemize}			
\item \textbf{\analyst:} è responsabile per le attività di analisi, in particolare:
	\begin{itemize}
	\item Comprendere la natura e la complessità del problema;
	\item Redigere lo \emph{Studio di Fattibilità} e l'\emph{Analisi dei Requisiti}.
	\end{itemize}
\item \textbf{\verifier:} è responsabile delle attività di verifica in particolare deve:
	\begin{itemize}
	\item Assicurare che lo svolgimento delle attività sia conforme alle norme stabilite nel documento \emph{Norme di Progetto};
	\item Controllare la conformità di ogni documento alle norme stabilite;
	\item Redigere il \emph{Piano di Qualifica} assieme all'\administrator.
	\end{itemize}
\item \textbf{\designer:} è responsabile per le attività che riguardano la progettazione in particolare deve:
	\begin{itemize}
	\item Presentare una soluzione comprensibile e attuabile motivandone le scelte fatte;
	\item Attuare scelte su aspetti progettuali che attribuiscano al prodotto soluzioni note e ottimizzate;
	\item Attuare scelte su aspetti tecnologici che rendono il prodotto manutenibile in modo semplice;
	\item Redigere la \emph{Specifica Tecnica} e la \emph{Definizione di Prodotto} oltre alle sezioni riguardanti le metriche di verifica della programmazione presente nel \emph{Piano di Qualifica}.
	\end{itemize}
\item \textbf{\programmer:} è responsabile delle attività riguardanti la codifica oltre che delle componenti necessarie per eseguire le prove di verifica e validazione del codice. Inoltre deve:
	\begin{itemize}
	\item Implementare come naturale conseguenza, le soluzioni descritte dal \designer{} che porteranno poi alla realizzazione del prodotto;
	\item Scrivere il codice che dovrà essere necessariamente: documentato, versionato, manutenibile e che rispetti le norme stabilite per la scrittura del codice presenti nelle \emph{Norme di Progetto};
	\item Implementare i test sul codice scritto utili per eseguire prove di verifica e validazione di quanto fatto;
	\item Redigere il \emph{Manuale Utente} e produrre documentazione del codice.
	\end{itemize}
\end{itemize}