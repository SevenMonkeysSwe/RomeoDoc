%%%%%%%%%%%%%%%%%%%%%%%%%%%%%%%%%%%%%%%%%%%%%%%
%Intoduzione
%Version: 0.0.1
%Author: Nicolò Bissacco(nickbissa@gmail.com)
%Creation Date: 2013-12-11
%Change Table
%Relase		Author			Date		Object of the changed
%%%%%%%%%%%%%%%%%%%%%%%%%%%%%%%%%%%%%%%%%%%%%%%%

\section{Introduzione}
\label{Introduzione}

	\subsection{Scopo del documento}
	\label{ScopoDelDocumento}
		Gli obiettivi del seguente documento sono molteplici:
		\begin{itemize}
			\item Specificare la pianificazione dei lavori dal gruppo \authorName{} sul progetto \project{};
			\item Pianificare le tempistiche delle attività da attuare;
			\item Preventivare e mettere a consuntivo le risorse necessarie ed utilizzate;
			\item Analizzare i possibili fattori di rischio ed escogitare opportune contromisure.
		\end{itemize}
	\subsection{Glossario}
		\label{Glossario}
		\glossIntro{}
	\subsection{Riferimenti}
	\label{Riferimenti}
		\subsubsection{Normativi}
		\label{RifNormativi}
			\begin{itemize}
				\item \textbf{Capitolato d'Appalto C3:} \project{} Medical Imaging Cluster Analysis Tool, \proposerName;\\				
				\url{http://www.math.unipd.it/~tullio/IS-1/2013/Progetto/C3p.pdf};
				
				\item \textbf{Vincoli di organigramma:}	\url{http://www.math.unipd.it/~tullio/IS-1/2013/Progetto/PD01b.html};
				
				\item \textbf{Norme di Progetto:} \NdP.		
			\end{itemize}
			
		\subsubsection{Informativi}
		\label{RifInformativi}
			\begin{itemize}
				\item \textbf{Software Engineering-Ian Sommerville-9th Edition(2010):} Part 4: Software Management;
				
				\item \textbf{Slide del Corso Ingegneria del Software modulo A:} \url{http://www.math.unipd.it/~tullio/IS-1/2013/}.
			\end{itemize}
	\subsection{Note sulle tabelle}
	\label{NoteSulleTabelle}
		Come riportato nelle \NdP, per aumentare la leggibilità le celle in cui il valore 0 (zero) non è significativo per la comprensione della tabella vengono lasciate vuote.
	
	\subsection{Ciclo di vita}
	\label{CicloDiVita}
		Come modello di ciclo di vita da applicare ai processi si è scelto il \textbf{modello incrementale}.\\
		Questo modello prevede:
		\begin{itemize}
			\item \textbf{Analisi e progettazione architetturale:} queste fasi non sono ripetute;\\
			I requisiti principali e l'architettura del sistema sono identificati e fissati completamente in modo da pianificare in modo corretto i cicli di incremento;
			
			\item \textbf{Progettazione di dettaglio, codifica e prove:} queste fasi sono realizzate in modo incrementale.\\
			Si prevedono due fasi di codifica permettendo il rilascio del software con le funzionalità obbligatorie il prima possibile e successivamente un nuovo rilascio con le funzionalità opzionali e altri miglioramenti.
		\end{itemize}			
	
	\subsection{Scadenze}
	\label{Scadenze}
		Di seguito le scadenze che il gruppo \authorName{} ha deciso di rispettare e sulle quali si baserà la pianificazione del progetto:
		\begin{itemize}
			\item \textbf{Revisione Requisiti (RR):} 2014-01-09;
			\item \textbf{Revisione di Progetto (RP):} 2014-02-10;
			\item \textbf{Revisione di Qualifica (RQ):} 2014-03-17;
			\item \textbf{Revisione di Accettazione (RA):} 2014-07-02.
		\end{itemize}