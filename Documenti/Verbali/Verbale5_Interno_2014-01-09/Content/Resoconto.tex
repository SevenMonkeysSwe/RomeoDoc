\section{Riassunto della riunione}
\label{riassunto}
\paragraph{}
La riunione è iniziata discutendo il primo punto all'ordine del giorno: la presentazione della \textit{Revisione dei Requisiti}. Durante il confronto su questo punto, è emersa l'evidenza che questo aspetto della consegna, è stato il più debole. Data la volontà del gruppo di migliorare questo aspetto del proprio lavoro, sono stati individuati alcuni punti deboli, che attraverso degli accorgimenti correttivi, potrebbero incrementarne l'efficacia in futuro.\\
In primo luogo, è stata valutata la presentazione in se stessa. Il gruppo ha concordato nel ritenere le slides troppo numerose per una presentazione di questa portata. Sebbene l'impatto visivo e comunicativo sia stato adeguato, la numerosità e l'assenza di un'organizzazione (intesa come filo logico del discorso) efficacie delle slides, probabilmente porta l'ascoltatore a "perdersi" all'interno della presentazione.\\
In secondo luogo, i contenuti esposti sono stati evidentemente insufficienti. Il gruppo si impegnerà in futuro a concentrare l'attenzione sui punti fondamentali che riguardano la presentazione. Inoltre, è emerso che un approfondimento più dettagliato sulle attività svolte concretamente dal gruppo, potrebbe giovare. Questi contenuti andranno, naturalmente, inseriti nelle slides.\\
Infine, è stata valutata l'organizzazione e la tempistica con cui è stata preparata la presentazione. Si è convenuto che, anticipando l'inizio dei lavori della presentazione, ci sarebbe, in futuro, più tempo per provare e perfezionare l'esposizione. Compatibilmente con le tempistiche di sviluppo del progetto, si cercherà di iniziare il lavoro di preparazione almeno 10 giorni lavorativi prima della data che verrà designata.

\paragraph{}
La seconda parte della riunione è stata dedicata a comprendere il punto della situazione dello stato dei lavori. In particolare, si è preso visione dal \textit{Piano di Progetto}, di quali ruoli, i vari componenti del gruppo, dovranno ricoprire in questa seconda macro-fase. In base a questa ripartizione, si è deciso di procedere in un primo luogo alla correzione dei documenti presentati alla \textit{Revisione dei Requisiti}. Successivamente, il lavoro si concentrerà sulla progettazione del software e alla conseguente stesura del documento di \textit{Specifica Tecnica}, che ne delinea i tratti macroscopici.