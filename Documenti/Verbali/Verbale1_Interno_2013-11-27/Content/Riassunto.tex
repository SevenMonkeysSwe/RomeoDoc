%%%%%%%%%%%%%%%%%%%%%%%%%%%%%%%%%%%%%%%%%%%%%%%%%%%%%%%%%%%%%%%%%%%%%%%%%%%%%%%%%%%%%%%%%%%%%%%%%%%%%%%%%%%%%%%%%%
% File: Riassunto.tex
% Current Version: 0.0.1
% Created: 2013-11-27
% Author: Feltre Beatrice
% Email: beatrice.feltre@gmail.com
%----------------------------------------------------------------------------------------------------------------
% Modification History:
% Version		Author	 			Date			Action
% 0.0.1			Feltre Beatrice		2013-11-27		Creazione e stesura file.
%%%%%%%%%%%%%%%%%%%%%%%%%%%%%%%%%%%%%%%%%%%%%%%%%%%%%%%%%%%%%%%%%%%%%%%%%%%%%%%%%%%%%%%%%%%%%%%%%%%%%%%%%%%%%%%%%%
\section{Riassunto della riunione}
\label{riassunto}
Il primo punto all'ordine del giorno di questa prima riunione, era la scelta del nome da assegnare al gruppo. Ogni componente, nei giorni precedenti, ha preparato delle proposte che ha esposto agli altri componenti. Dopo due votazioni, si è scelto a maggioranza di nominare il gruppo: \authorName. Successivamente, si è discusso a proposito del logo e dell'impaginazione dei documenti formali da presentare ai committenti.
\\Nel secondo punto, si è discusso di come distribuire i ruoli ai vari componenti e di come gestirne la rotazione. La specifica di questo aspetto, verrà esposta esaustivamente nel documento \textit{Piano di Progetto v1.0.0}.
\\Il terzo punto prevedeva una discussione sommaria su quali strumenti e servizi utilizzare durante il progetto. Si è deciso quindi di appoggiarsi a GitHub\glossario{} per avere un servizio gratuito di repository\glossario{}. Inoltre è stato scelto di utilizzare i servizi offerti da \textit{Google}, per creare una e-mail ufficiale del gruppo (GMail) e per gestire la comunicazione informale tra i componenti (Google Hangout\glossario{}). \`E stato valutato Astah\glossario{}, come software per generare i diagrammi UML\glossario{} necessari e il software GanttProject\glossario{} per produrre i diagrammi Gantt\glossario{}. I dettagli di questo aspetto verranno discussi approfonditamente nel documento \textit{Norme di Progetto v1.0.0}.
\\L'ultimo punto all'ordine del giorno, consisteva nella scelta definitiva del capitolato da svolgere. Dopo aver precedentemente scartato i capitolati \textbf{C4} e \textbf{C5}, alla luce delle informazioni raccolte durante gli incontri informali avuti con gli altri proponenti, si è passato ad una votazione per decidere. La maggioranza, ha scelto di sviluppare il progetto esposto nel capitolato \textbf{C3}. Le considerazioni dettagliate che hanno portato a questa decisione, verranno esposte nel documento \textit{Studio di Fattibilita v1.0.0}.
\\A seguito di questa decisione, sono stati contattati nuovamente i proponenti per fissare un primo appuntamento per approfondire gli aspetti del capitolato.