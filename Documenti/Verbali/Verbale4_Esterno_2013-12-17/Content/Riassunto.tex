%%%%%%%%%%%%%%%%%%%%%%%%%%%%%%%%%%%%%%%%%%%%%%%%%%%%%%%%%%%%%%%%%%%%%%%%%%%%%%%%%%%%%%%%%%%%%%%%%%%%%%%%%%%%%%%%%%
% File: Riassunto.tex
% Current Version: 0.0.1
% Created: 2013-12-17
% Author: Bissacco Nicolò
% Email: nickbissa@gmail.com
%----------------------------------------------------------------------------------------------------------------
% Modification History:
% Version		Author	 			Date			Action
% 0.0.1			Bissacco Nicolò		2013-12-17		Creazione e stesura file.
%%%%%%%%%%%%%%%%%%%%%%%%%%%%%%%%%%%%%%%%%%%%%%%%%%%%%%%%%%%%%%%%%%%%%%%%%%%%%%%%%%%%%%%%%%%%%%%%%%%%%%%%%%%%%%%%%%
\section{Riassunto}
\label{riassunto}
Durante l'analisi dei requisiti, è sorto un dubbio riguardo alla flessibilità del prodotto software finale. Per questo, sono stati contattati i proponenti, chiedendo se concordavano nell'inserire un requisito opzionale in più.
\\La questione stava nel permettere o meno, all'utente, di selezionare quali Subject\glossario{} appartenenti ad un gruppo, potessero essere processati dal Protocol\glossario{}. In sostanza, dopo aver fatto creare e salvare un gruppo di Subject\glossario{} all'utente, è stato considerato poco flessibile impedire a quest'ultimo la scelta di quali Subject\glossario{} processare.
\\Dopo aver esposto le precedenti considerazioni ai proponenti, quest'ultimi hanno accettato di inserire questo requisito opzionale nell'analisi dei requisiti. Di seguito, la mail di risposta, ricevuta in data 2013-12-17, che approva l'autorizzazione:
\\ \\ \lq\lq \textit{Ciao ragazzi,}
\\ \\\textit{per noi va bene. Nel senso, la modifica dei requisiti è prevista dal capitolato, ma preferiamo che per venire incontro a tutti ci sia accordo comune su tali modifiche, quando sono proposte da uno dei gruppi. Per questo motivo non modifichiamo il capitolato.}
\\ \\\textit{In ogni caso per noi va bene che lo inseriate nella vostra analisi.}
\\ \\\textit{Una precisazione: dare la possibilità all'utente di analizzare un sottoinsieme di Subjects\glossario{} dal gruppo, implica anche la possibilità a questo punto di poter aggiungere o eliminare Subjects\glossario{} in un secondo momento rispetto alla creazione del gruppo. Rimane implicito che l'eliminazione non deve comportare l'eliminazione di eventuali risultati.} 
\\ \\\textit{Attenzione che il secondo punto non è una nostra richiesta, ma solo una precisazione. Potete decidere di includere o meno questo aspetto nella vostra analisi.} 
\\ \\\textit{Ciao,
Gaia} \rq\rq