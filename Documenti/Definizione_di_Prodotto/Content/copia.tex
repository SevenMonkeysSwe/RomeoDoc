\section{Diagrammi di sequenza}
\label{diagrammi_seq}

Di seguito sono riportati i diagrammi di sequenza delle operazioni principali di \project{}.

\subsection{Creazione di un nuovo Subject}
\label{creazione_subject}
Il diagramma seguente, descrive la sequenza di operazioni che vengono effettuate quando si procede all'inserimento di un nuovo Subject\g{}. La sequenza delle operazioni viene scatenata dall'utente quando seleziona, dalla finestra principale di \project{} (\textit{WelcomePage}), il comando \textit{New Subject}, tramite l'apposito pulsante. Vengono successivamente effettuate le seguenti operazioni:
\begin{enumerate}
	\item Viene invocato il metodo \textit{CreateNewSubject()} del \textit{WelcomeController};
	\item Il \textit{WelcomeController} richiama lo slot \textit{SlotNewSubject()} del \textit{MainWindowController};
	\item Il \textit{MainWindowController} crea una nuova istanza della vista \textit{NewSubjectView} che permette all'utente di interagire col sistema per inserire il Subject\g{}. In seguito \textit{MainWindowController} crea un'istanza di \textit{NewSubjectController};
	\item \textit{MainWindowController} invoca il proprio metodo \textit{RegisterToSystem}, il quale sostituisce la vista corrente (\textit{WelcomePage}) con la nuova vista \textit{NewSubjectView}, cancellandola. In seguito registra anche il nuovo controller \textit{NewSubjectController} invocando il metodo \textit{addController} di \textit{ControlManager}, cancellando anche il controller relativo alla finestra principale;
\end{enumerate}
L'utente a questo punto visualizza la vista che permette l'inserimento delle informazioni del \Subject{}, quali: \textit{Nome}, \textit{Tipo}, \textit{Immagine/Video} e \textit{Maschera}. L'aggiunta dei file relativi all'immagine e alla maschera prevede le seguenti informazioni:
\begin{enumerate}
	\item Emissione del signal\g{} \textit{addImage()/addMask()}, raccolto dal \textit{NewSubjectController} che procede lanciando lo slot \textit{slotAddImage()/slotAddMask()}; quest'ultimo crea una finestra di dialogo che permette all'utente di selezionare il file relativo all'immagine;
	\item Dopo aver caricato il file, il controller invocherà il metodo \textit{setImagePath()/setMaskPath()}, indicando il percorso del file selezionato.
\end{enumerate}
Una volta inserite le informazioni sul \Subject, l'utente potrà salvare il Subject\g{} premendo il pulsante \textit{Save}. La pressione del pulsate provoca le seguenti operazioni:
\begin{enumerate}
	\item Emmissione del signal\g{} \textit{saveObject()}, raccolto dal \textit{NewSubjectController} che procede lanciando lo slot \textit{slotSaveSubject}, il quale salva il Subject\g{} appena creato.
\end{enumerate}

Al fine di agevolare la lettura del diagramma sono state omesse le operazioni di aggiornamento della \textit{StatusBar} di \textit{MainWindow}.