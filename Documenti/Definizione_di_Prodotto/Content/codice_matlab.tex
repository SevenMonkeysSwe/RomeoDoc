\subsection{Codice Matlab}
\label{codicematlab}

Il proponente fornisce alcuni file di codice matlab, che mostrano gli algoritmi di ogni singola feature extractor\g{} applicata su matrici matlab che rappresentano immagini bidimensionali.

\subsubsection{Standard Deviation Feature}
\label{stdcode}
Questa funzione matlab rappresenta l'algoritmo per calcolare il valore di un pixel 3D per la feature di Standard Deviation:

\lstloadlanguages{MATLAB}
\begin{lstlisting}

% std returns the standard deviation value
% within the window (in this example 3D window)
function std=Feat_p1_std(data,size)
% INPUT: 
% data: window data
% size: window size
nvol = size^3;
val=Feat_p1_mean(data,size);
ct = 0;
for r=1:size
	for c=1:size
		for s=1:size
			ct=ct+(data(r,c,s)-val)^2;
		end
	end
end
std = sqrt(ct/nvol);
end
\end{lstlisting} 

\subsubsection{Skewness Feature}
\label{skewnesscode}
Questa funzione matlab rappresenta l'algoritmo per calcolare il valore di un pixel 3D per la feature Skewness:

\begin{lstlisting}
% skewness returns the skewness 
%within the window (in this example 3D window)
function skew=Feat_p1_skew(data,size)
% INPUT: 
% data: window data
% size: window size
nvol = size^3;
val=Feat_p1_mean(data,size);
std=Feat_p1_std(data,size);
ct = 0;
for r=1:size
	for c=1:size
		for s=1:size
			ct=ct+(data(r,c,s)-val)^3;
		end
	end
end
ct2 = ct/nvol;
skew = ct2/(std^3);
end
\end{lstlisting}

\subsubsection{Mean Feature}
\label{meancode}
Questa funzione matlab rappresenta l'algoritmo per calcolare il valore di un pixel 3D per la feature Mean:

\begin{lstlisting}
% mean returns the mean value 
%within the window (in this example 3D window)
function val=Feat_p1_mean(data,size)
% INPUT: 
% data: window data
% size: window size
nvol = size^3;
ct = 0;
for r=1:size
	for c=1:size
		for s=1:size
			ct=ct+data(r,c,s);
		end
	end
end
val = ct/nvol;
end
\end{lstlisting}

\subsubsection{Kurtosis Feature}
\label{kurtosiscode}
Questa funzione matlab rappresenta l'algoritmo per calcolare il valore di un pixel 3D per la feature Kurtosis:

\begin{lstlisting}
% kurtosis returns the kurtosis 
%within the window (in this example 3D window)
function kurt=Feat_p1_kurt(data,size)
% INPUT: 
% data: window data
% size: window size
nvol = size^3;
val=Feat_p1_mean(data,size);
std=Feat_p1_std(data,size);
ct = 0;
for r=1:size
	for c=1:size
		for s=1:size
			ct=ct+(data(r,c,s)-val)^4;
		end
	end
end
ct2 = ct/nvol;
kurt = ct2/(std^4)-3;
end
\end{lstlisting}

\subsubsection{Contrast Feature}
\label{contrastcode}
Questa funzione matlab rappresenta l'algoritmo per calcolare il valore di un pixel 2D per la feature Contrast:
\begin{lstlisting}
% contrast returns the contrast within the window
%evaluated from GLCM (sempre 2D)
function contr=Feat_p1_contr(data,size)
% INPUT: 
% data: GLCM
% size: GLCM size
ct = 0;
for i=1:size
	for j=1:size
		ct = ct+data(i,j)*(i-j)^2;
	end
end
contr = ct;
end
\end{lstlisting}
\subsubsection{Correlation Feature}
\label{correlationcode}
Questa funzione matlab rappresenta l'algoritmo per calcolare il valore di un pixel 2D per la feature Correlation:
\begin{lstlisting}
% corr returns the correlation within the window evaluated from GLCM (sempre 2D)
function corr=Feat_p1_corr(data,size)
% INPUT: 
% data: GLCM
% size: GLCM size
% FORMULA: 
% Mi and Mj are vectors 
Mi = meanIndexI(data,size);
Si = stdIndexI(data,size,Mi);

Mj = meanIndexJ(data,size);
Sj = stdIndexJ(data,size,Mj);

ct2 = 0;
for i=1:size
	for j = 1:size
		ct1 = data(i,j) * (i-Mi(i))*(j-Mj(j));
		ct2 = ct2 + ct1 / (Si(i) * Sj(j));
	end
end
	
corr = ct2;
	
%-------------------------------------------------
function S = stdIndexI(data,size,m)

ct = 0;
for i=1:size
	for j=1:size
		ct = ct + data(i,j)*(i-m(i))^2;
	end
	S(i) = sqrt(ct);
	ct = 0;
end

%--------------------------------------------------
function M = meanIndexI(data,size)

ct = 0;
for i=1:size
	for j=1:size
		ct = ct+ i * data(i,j);
	end
	M(i) = ct;
	ct = 0;
end

end%---------------------------------------------

function S = stdIndexJ(data,size,m)

ct = 0;
for j=1:size
	for i=1:size
		ct = ct + data(i,j)*(j-m(j))^2;
	end
	S(j) = sqrt(ct);
	ct = 0;
end

%----------------------------------------------
function M = meanIndexJ(data, size)

ct = 0; 
for j=1:size
	for i=1:size
		ct = ct + j*data(i,j);
	end
	M(j) = ct;
	ct = 0;
end

end
\end{lstlisting}
\subsubsection{Energy Feature}
\label{energycode}
Questa funzione matlab rappresenta l'algoritmo per calcolare il valore di un pixel 2D per la feature Energy:
\begin{lstlisting}
% energy returns the energy within 
%the window evaluated from GLCM (sempre 2D)
function energy=Feat_p1_energy(data,size)
% INPUT: 
% data: GLCM
% size: GLCM size
ct = 0;
for i=1:size
	for j=1:size
		ct = ct+data(i,j)^2;
	end
end
energy = ct;
end
\end{lstlisting}
\subsubsection{Entropy Feature}
\label{entropycode}
Questa funzione matlab rappresenta l'algoritmo per calcolare il valore di un pixel 2D per la feature Entropy:
\begin{lstlisting}
% entropy returns the entropy within 
%the window evaluated from GLCM (sempre 2D)
function entropy=Feat_p1_entropy(data,size)
% INPUT: 
% data: GLCM
% size: GLCM size
ct = 0;
for i=1:size
	for j=1:size
		ct = ct+data(i,j)*ln(data(i,j));
	end
end
entropy = -ct;
end
\end{lstlisting}
\subsubsection{Homogeneity Feature}
\label{homogeneitycode}
Questa funzione matlab rappresenta l'algoritmo per calcolare il valore di un pixel 2D per la feature Homogeneity:
\begin{lstlisting}
% homogeneity returns the homogeneity within
% the window evaluated from GLCM (sempre 2D)
function homo=Feat_p1_homo(data,size)
% INPUT: 
% data: GLCM
% size: GLCM size
ct = 0;
for i=1:size
	for j=1:size
		ct = ct+data(i,j)*(1/(1+abs(i-j)));
	end
end
homo = ct;
end
\end{lstlisting}

\subsubsection{Area Under the Curve Feature}
\label{auccode}
Questa funzione matlab rappresenta l'algoritmo per calcolare il valore di un pixel 2D per la feature Area Under the Curve in un video 2D:
\begin{lstlisting}
% AUC computes the area under the curve in an interval of frames 
function AUC=Clust_par_auc(data,nrow,posin,posend)
% INPUT: 
% data: data matrix
% nrow: nr of curves (nr of voxels)
% posin: index of first frame
% posend: index of last frame
AUC=zeros(nrow,1);
for i=1:nrow
    AUC(i)=trapz([posin:posend],data(i,posin:posend),2);
end
end
\end{lstlisting}
\subsubsection{Maximum Feature}
\label{maximumcode}
Questa funzione matlab rappresenta l'algoritmo per calcolare il valore di un pixel 2D per la feature Maximum in un video 2D:
\begin{lstlisting}
% max searches the maximum value in an interval of frames 
function maxval=Clust_par_max(data,nrow,posin,posend)
% INPUT: 
% data: data matrix
% nrow: nr of curves (nr of voxels)
% posin: index of first frame
% posend: index of last frame
maxval=zeros(nrow,1);
for i=1:nrow
maxval(i)=max(data(i,posin:posend));
end
end
\end{lstlisting}
\subsubsection{Minimum Feature}
\label{minimumcode}
Questa funzione matlab rappresenta l'algoritmo per calcolare il valore di un pixel 2D per la feature Minimum in un video 2D:
\begin{lstlisting}
% min searches the min value in an interval of frames 
function val=Clust_par_val(data,nrow,posin,posend)
% INPUT: 
% data: data matrix
% nrow: nr of curves (nr of voxels)
% posin: index of first frame
% posend: index of last frame
minval=zeros(nrow,1);
for i=1:nrow
minval(i)=min(data(i,posin:posend));
end
end
\end{lstlisting}
\subsubsection{Mean Dynamic Feature}
\label{meandynamiccode}
Questa funzione matlab rappresenta l'algoritmo per calcolare il valore di un pixel 2D per la feature Mean Dynamic in un video 2D:
\begin{lstlisting}
% mean returns the mean value in an interval of frames 
function val=Clust_par_mean(data,nrow,posin,posend)
% INPUT: 
% data: data matrix
% nrow: nr of curves (nr of voxels)
% posin: index of first frame
% posend: index of last frame
meanval=zeros(nrow,1);
for i=1:nrow
meanval(i)=average(data(i,posin:posend));
end
end
\end{lstlisting}