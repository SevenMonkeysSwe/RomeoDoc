\section{Introduzione}
\label{introduzione}
	\subsection{Scopo del documento}
	\label{scopo_doc}
	Il seguente documento ha lo scopo di definire nel dettaglio la struttura del sistema \project{}, approfondendo quanto già riportato nel documento \emph{Specifica Tecnica v2.0.0}. Tale documento fornisce una struttura dettagliata e completa che viene utilizzata dai \emph{programmatori} per le attività di codifica.
	\subsection{Scopo del prodotto}
	\label{scopo_prodotto}
	\scopoProd{}
	\subsection{Glossario}
	\label{glossario}
	\glossIntro{}
	\subsection{Riferimenti}
	\label{riferimenti}
		\subsubsection{Normativi}
		\label{normativi}
		\begin{itemize}
		\item \textbf{Specifica Tecnica:} \emph{Specifica Tecnica v3.0.0};
		\item \textbf{Analisi dei Requisiti:} \emph{Analisi dei Requisiti v4.0.0};
		\item \textbf{Norme di Progetto:} \emph{Norme di Progetto v4.0.0}.
		\end{itemize}
		\subsubsection{Informativi}
		\begin{itemize}
		\item \textbf{Documentazione Qt\g{} per Signal\g{} e Slot\g{}} \\ \url{http://qt-project.org/doc/qt-5.0/qtcore/signalsandslots.html};
		\item \textbf{Documentazione ITK\g{}} \\ \url{http://itk.org/Doxygen/html/};
		\item \textbf{Documentazione OpenCv} \\ \url{http://docs.opencv.org/};
		\item \textbf{Documentazione QtAssistant} \\ \url{http://qt-project.org/doc/qt-4.8/assistant-manual.html};
		\item \textbf{Glossario:} \emph{Glossario v3.0.0}.
		\end{itemize}