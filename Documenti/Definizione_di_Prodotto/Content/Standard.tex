\section{Standard di progetto}
\label{standard_prog}
\subsection{Standard di progettazione architetturale}
\label{progettazioneStandard}
Gli standard della progettazione architetturale sono definiti nella \emph{Specifica Tecnica v2.0.0}.

\subsection{Standard di documentazione del codice}
\label{documentazioneStandard}
Gli standard per la documentazione del codice sono definiti nelle \emph{Norme di Progetto v3.0.0}.
\subsection{Standard di denominazione delle entità e relazioni}
\label{denominazioneStandard}
Tutti gli elementi definiti, package\g{}, classi, metodi o attributi, devono avere una denominazione chiara e autoesplicativa. Qualora il nome risulti essere lungo è preferibile la chiarezza alla lunghezza.
Le abbreviazione sono ammesse solo se:
\begin{itemize}
\item immediatamente comprensibili;
\item non ambigue.
\end{itemize}

\subsection{Standard di programmazione}
\label{programmazioneStandard}

Gli standard di programmazione sono definiti e descritti nelle \emph{Norme di Progetto v3.0.0}.

\subsection{Strumenti di lavoro}
\label{strumentiLavoro}

Gli strumenti da adottare e le procedure da seguire per utilizzarli correttamente durante la realizzazione del prodotto software sono definiti nelle \emph{Norme di Progetto v3.0.0}.