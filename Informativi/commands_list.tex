%%%%%%%%%%%%%%%%%%%%%%%%%%%%%%%%%%%%%%%
% File: command_list.tex
% Created: 2013 November 27
% Author: Magnabosco Nicola
% Email: nick.magnabosco@gmail.com
%%%%%%%%%%%%%%%%%%%%%%%%%%%%%%%%%%%%%%%
\documentclass[10pt]{article}
	\usepackage[italian]{babel}
	\usepackage[utf8]{inputenc}
	\usepackage[T1]{fontenc}
	\usepackage{geometry}
	\usepackage{graphicx}%package per la gestione delle immagini
	\geometry{a4paper}
\newcommand{\version}{v2.0.0}
\title{LISTA DI COMANDI USATI NEI DOCUMENTI LaTeX \version}
\author{Nicola Magnabosco}
\begin{document}
\maketitle
	\paragraph{Comandi da Modificare all'interno del \textbf{template}}
	\begin{itemize}
			\item \textbf{documentName} stampa il nome del documento
			\item \textbf{currentVersion }stampa la versione corrente del documento
			\item \textbf{redactionDate} stampa il giorno di compilazione del documento (fa uso del comando "today")
			\item \textbf{editors} stampa la lista dei redattori
			\item \textbf{verifiers} stampa la lista dei verificatori
			\item \textbf{approved} stampa il nome di chi ha approvato il documento
			\item \textbf{usage} stampa il campo d'uso del documento, interno o esterno
			\item \textbf{distributionList} stampa la lista di distribuzione del documento
			\item \textbf{sommario} inserisce la descrizione dello scopo del documento
			\item \textbf{glossario} inserisce una \textit{G} maiuscola in pedice
	\end{itemize}
	\paragraph{Comandi Standard all'interno del file commands.tex}
	\begin{itemize}
		\item \textbf{project} stampa "ROMEO"
		\item \textbf{email} stampa l'url "7monkeys.swe@gmail.com"
		\item \textbf{authorName} stampa "7Monkeys"
		\item \textbf{administrator} stampa in \textit{emph} "Amministratore di Progetto"
		\item \textbf{projectManager} stampa in \textit{emph} Responsabile di Progetto
		\item \textbf{analyst} stampa in \textit{emph} Analista
		\item \textbf{programmer} stampa in \textit{emph} Programmatore
		\item \textbf{designer} stampa in \textit{emph} Progettista
		\item \textbf{verifier} stampa in \textit{emph} Verificatore
		\item \textbf{proposerName} stampa in \textit{emph} Departement of Information Engineering (DEI)
		\item \textbf{commitName} stampa in \textit{emph} Prof. Tullio Vardanega
		\item \textbf{logo}  include dell'immagine "big-logo"
		
	\end{itemize}
\end{document}